\documentclass[svgnames]{amsart}
\usepackage[paperwidth=6in, paperheight=8in, top = 20mm, bottom = 18mm, left=10mm, right = 10mm]{geometry}

\usepackage{amsmath, amsfonts, amssymb, amsthm}
\usepackage{graphicx, tikz}
\usepackage{mathtools, physics}

\usepackage{enumitem}
\setlist[enumerate,1]{label=\arabic*.}

\usepackage[default,bold]{sourceserifpro}
\usepackage[T1]{fontenc}
\usepackage[]{fouriernc}

\usepackage{xcolor}
\usepackage{scrlayer-scrpage}
\ohead{\color{blue!35!black} \scshape VM}
\cfoot*{\scriptsize\pagemark}

\renewcommand{\th}{\textsuperscript{th}}

\setlength{\parindent}{0pt}

\title[]{Probability and Statistics -- Problem Set 8}

\DeclareMathOperator{\Prob}{P}
\DeclareMathOperator{\EV}{\mathbb E}
\DeclareMathOperator{\Var}{\mathbb V}


\begin{document}
\maketitle
\begin{enumerate}[leftmargin=*, itemsep=0.3em]
\item Compute the moment generating function of the discrete random variable $X$ with pmf $f(x) = \dfrac{1}{2^x}$, $x = 1, 2, \ldots$.

\item If $0 < p < 1$ and $q = 1 - p$, find the mgf of the random variable $X$ with pmf $f(x) = p^{x - 1} q$, $x = 1, 2, \ldots$, and hence find $E[X]$ and $V[X]$.

\item If $E[X^n] = 2^n (n + 1)!$, $n = 0, 1, \ldots$, then find the mgf of $X$.

\item Let $X$ be a random variable having pdf $f(x) = a e^{-a(x - b)}$, $x \ge b$, where $a > 0$. Using the mgf of $X$, determine its mean and variance.

\item If $X$ is a random variable with pdf $f(x) = e^{-2|x|}$, find the mgf of $X$, and hence compute $E[X]$ and $V[X]$.

\item If $X \sim U[a, b]$, compute $E[X^n]$ using $M_X(t)$. Hence show that if $X \sim U[-a, a]$, then $E[X^{2n}] = \dfrac{a^{2n}}{2n + 1}$.

\item If $M_{X_1}(t) = e^{3t + 2t^2}$, $M_{X_2}(t) = e^{5t + 18t^2}$, $M_{X_3}(t) = e^{4t + 8t^2}$, then find the pdf of $Y = 2X_1 + 3X_2 + 4X_3$, given that $X_1$, $X_2$, and $X_3$ are independent.

\item If $X \sim N(0, 2)$, then find the moment generating function of $Y = \frac{X^2}{2}$.

\item Find the mean of $X$, given that its mgf is $M_X(t) = e^{2(e^t - 1)}$.

\item Find the variance of $X$, given that its mgf is $M_X(t) = \left(\frac 3 4 + \frac{e^t}{4}\right)^{20}$.

\item Show that if $X$ and $Y$ are independent Poisson variate with means $\lambda$ and $\mu$ respectively, then $X + Y$ is a Poisson variate with mean $\lambda + \mu$.

\item Show that if $X_i \sim N(\mu_i, \sigma_i^2)$, $i = 1, \ldots, n$ are $n$ independent normal variates, and $a_i$, $i = 1, \ldots, n$ are constants, then $X = \sum_{i=1}^{n} a_i X_i \sim N(\mu, \sigma^2)$ where $\mu = \sum_{i = 1}^{n} a_i \mu_i$ and $\sigma = \sqrt{\sum_{i = 1}^n a_i^2 \sigma_i^2}$. Also determine the distribution of $\overline X = \frac X n$.

\item Show that if $Z_i \in N(0, 1)$, $i = 1, \ldots, n$ are $n$ independent standard normal variates, then $\sum_{i = 1}^n Z_i^2 \sim \chi^2_n$.

\end{enumerate}

\end{document}