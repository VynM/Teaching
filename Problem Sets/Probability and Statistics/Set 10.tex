\documentclass[svgnames]{amsart}
\usepackage[paperwidth=6in, paperheight=8in, top = 20mm, bottom = 18mm, left=10mm, right = 10mm]{geometry}

\usepackage{amsmath, amsfonts, amssymb, amsthm}
\usepackage{graphicx, tikz}
\usepackage{mathtools, physics}

\usepackage{enumitem}
\setlist[enumerate,1]{label=\arabic*.}

\usepackage[default,bold]{sourceserifpro}
\usepackage[T1]{fontenc}
\usepackage{eulervm}

\usepackage{xcolor}
\usepackage{scrlayer-scrpage}
\ohead{\color{blue!35!black} \scshape VM}
\cfoot*{\scriptsize\pagemark}

\renewcommand{\th}{\textsuperscript{th}}

\setlength{\parindent}{0pt}

\title[]{Probability and Statistics -- Problem Set 10}

\DeclareMathOperator{\Prob}{P}
\DeclareMathOperator{\EV}{\mathbb E}
\DeclareMathOperator{\Var}{\mathbb V}


\begin{document}
\maketitle
\begin{enumerate}[leftmargin=*, itemsep=2mm]
\item If $X_1, \ldots, X_n$ is an independent sample of size $n$ from a population defined by the pmf $f(x) = \theta^x (1 - \theta)^{1 - x}$, $x = 0, 1$, where $0 < \theta < 1$, find the maximum likelihood estimator for $\theta$.

\item Find the maximum likelihood estimator for $\theta$ in each of the following cases:
\begin{enumerate}
\item $f(x; \theta) = \theta x^{\theta - 1}$, $0 < x < 1$, where $\theta > 0$.
\item $f(x; \theta) = \theta e^{-\theta x}$, $x > 0$, where $\theta > 0$.
\item $f(x; \theta) = e^{-(x - \theta)}$, $x \ge \theta$, where $\theta$ is any real number.
\item $f(x; \theta) = \frac 1 2 e^{-|x - \theta|}$, where $\theta$ is any real number.
\item $f(x; \theta) = \dfrac {2x} {\theta^2}$, $0 \le x \le \theta$, where $\theta > 0$.
\end{enumerate}

\item Find the mles of $\theta_1$ and $\theta_2$ in each of the following cases:
\begin{enumerate}
\item $X \sim N(\theta_1, \theta_2)$
\item $f(x; \theta_1, \theta_2) = \dfrac{1}{\theta_2} e^{- \frac{(x - \theta_1)}{\theta_2}}$, $x > \theta_1$
\end{enumerate}
where $\theta_1$ is any real number and $\theta_2 > 0$.

\item Find the mle of $\theta$ in each of the following cases:
\begin{enumerate}
\item  $X \sim B(m, \theta)$.
\item $X \sim \mathcal P(\theta)$.
\item $X \sim N(\mu, \theta)$.
\end{enumerate}

\item Let $X \sim U[0, \theta]$, where $\theta > 0$. Compute the mle for $\theta$.

\item Show that the sample mean is an unbiased and consistent estimator of population mean.

\item Show that sample variance is a biased estimator of the population variance.

\item Show that $Y = \frac{1}{n - 1} \sum_{i = 1}^n (X_i - \overline X)^2$ is an unbiased estimator of population variance.

\item If $X \sim N(0, \theta)$, show that $Y = \frac 1 n \sum_{i=1}^n X_i^2$ is an unbiased estimator of $\theta$ and has variance $\frac {2\theta^2} n$.

\item Show that $\overline X$ is an unbiased estimator of $\theta$ if the pdf of $X$ is $f(x; \theta) = \dfrac 1 \theta e^{-\frac x \theta}$, $x > 0$, where $\theta > 0$. Also show that $\overline X$ has variance $\dfrac {\theta^2} n$ and is therefore a consistent estimator.

\item Let $Y_n$ be an unbiased estimator of $\theta$, such that $V(Y_n) \to 0$ as $n \to \infty$. Then show that $Y_n$ is consistent.

\item Let $Y_1$ and $Y_2$ be two independent unbiased statistics for $\theta$ such that the variance of $Y_1$ is twice that of $Y_2$. Find the constants $k_1$ and $k_2$ such that $Z = k_1 Y_1 + k_2 Y_2$ is an unbiased statistic for $\theta$ with minimum possible variance for such a linear combination.

\item Find the same size $n$ such that $\Prob[\overline X - 1 < \mu < \overline X + 1] = 0.9$, given that $X \sim N(\mu, 9)$.

\item If the observed value of the mean of a sample of size $20$ from a population having distribution $N(\mu, 80)$ is $\overline x = 81.2$, find a $95$ percent confidence interval for the population mean.

\item If a random sample of size $17$ from a normal distribution $N(\mu, \sigma^2)$ yields $\overline x = 4.7$ and $s^2 = 5.76$, determine a $90$ percent confidence interval for $\mu$.

\item A random sample of size from the distribution $N(\mu, \sigma^2)$ yields $s^2 = 7.63$. Determine a $95$ percent confidence interval for $\sigma^2$.

\item Find an approximate $95$ percent confidence interval for the mean of a population having variance $100$, if the sample size is $25$.

\item A random sample of size $15$ from a normal population with unknown mean and variance yields $\overline x = 3.2$ and $s^2 = 4.24$. Determine a $95$ percent confidence interval for $\sigma^2$.

\item If a sample of size $15$ from a population with distribution $N(\mu, \sigma^2)$ yields values $\sum_{i=1}^{15} X_i = 8.7$ and $\sum_{i = 1}^{15} X_i^2 = 27.3$, obtain a $95$ percent confidence interval for $\sigma^2$.

\item Suppose that $X \sim N(8, \sigma^2)$, and the observed values of a sample are of size $9$ from this population are $8.6$, $7.9$, $8.3$, $6.4$, $8.4$, $9.8$, $7.2$, $7.8$, and $7.5$. Construct a $90$ percent confidence interval for $\sigma^2$.

\item Suppose that $X \sim N(\mu, 4)$. If $\overline x = 78.3$ with $n = 25$, obtain a $99$ percent confidence interval for $\mu$.

\end{enumerate}

\end{document}