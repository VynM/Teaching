% !TeX root = Category Theory for Algebra II.tex

\title{\textbf{Category Theory for Algebra II}}

\date{\today}
\maketitle

\begingroup
\let\clearpage\relax
\tableofcontents
\endgroup

\clearpage

\renewcommand{\nomname}{List of Symbols}
\nomenclature{$\cat{Set}$}{Category of sets and functions}
\nomenclature{$\cat{Rel}$}{Category of sets and reltaions}
\nomenclature{$\cat{Grp}$}{Category of groups and group homomorphisms}
\nomenclature{$\cat{AbGrp}$}{Category of Abelian groups and group homomorphisms}
\nomenclature{$\cat{Mon}$}{Category of monoids and monoid homomorphisms}
\nomenclature{$\cat{SemiGrp}$}{Category of semigroups and semigroup homomorphisms}
\nomenclature{$\cat{Ring}$}{Category of unital rings and unital ring homomorphisms}
\nomenclature{$\cat{Rng}$}{Category of rngs and rng homomorphisms}
\nomenclature{$\cat{Vect}_F$}{Category of vector spaces over field $F$ and linear transformations}
\nomenclature{$\cat{FinVect}_F$}{Category of finite dimensional vector spaces over field $F$ and linear transformations}
\nomenclature{$R-\cat{Mod}$}{Category of left $R$-modules}
\nomenclature{$\cat{Mod}-R$}{Category of right $R$-modules}
\nomenclature{$\Ob(\cat C)$}{Collection of objects of category $\cat C$}
\nomenclature{$\Mor(\cat C)$}{Collection of morphisms of category $\cat C$}
\nomenclature{$\Hom_{\cat C}(X, Y)$}{Collection of morphisms from $X$ to $Y$ in category $\cat C$}
\nomenclature{$\cat C\opp$}{Opposite category of category $\cat C$}
\printnomenclature[10em]

\clearpage

%\section{Introduction}\label{sec:Intro}

\section{Examples of Categories}\label{sec:CatExamples}

Before formally defining a category, we shall look at some examples of familiar categories. A category consists of a collection of \newterm{obects} and a collection of \newterm{morphisms} satisfying certain conditions that will be discussed later.
\begin{enumerate}
\item The category $\cat{Set}$ has sets as objects and functions as morphisms.
\item The collection of groups and group homomorphisms forms a category denoted by $\cat{Grp}$.
\item The collection of vector spaces over a fixed field $F$, together with $F$-linear transformations forms the category $\cat{Vect}_F$.
\item The collection of sets and relations forms the category $\cat{Rel}$.
\end{enumerate}


\section{Initial and Final Objects}\label{sec:Terminals}

An \newterm{initial object} in a category $\cat C$ is an object $A$ such that for all objects $X$ of $\cat C$, there exists a unique morphism from $A$ to $X$. A \newterm{final object} in $\cat C$ is an object $A$ such that for all objects $X$ of $\cat C$, there exists a unique morphism from $X$ to $A$. A \newterm{zero object} is an object that is both initial and final.

\begin{Example}
\begin{enumerate}
\item The empty set $\varnothing$ is the initial object and any singleton set is a final object in $\cat{Set}$.
\item The trivial group is both the initial and final object in $\cat{Grp}$.
\item The zero-dimensional vector space is the zero object in $\cat{Vect}_F$.
\item The empty set is the zero object in $\cat{Rel}$.
\end{enumerate}
\end{Example}


\section{Categories}\label{sec:Categories}

A \newterm{category} $\cat C$ is a collection $\Ob(\cat C)$ of objects and, for every two objects $X$ and $Y$ a collection $\Hom(X, Y)$ of \newterm{morphisms} or \newterm{arrows}, together with operations
\begin{equation*}
    \circ \colon \Hom(Y, Z) \times \Hom(X, Y) \to \Hom(X, Z)
\end{equation*}
for all $X, Y, Z \in \Ob(\cat C)$, satisfying the following:
\begin{enumerate}[label=(C\arabic*)]
\item $\Hom(X_1, Y_1)$ and $\Hom(X_2, Y_2)$ are disjoint unless $X_1 = X_2$ and $Y_1 = Y_2$.
\item For each $X \in \Ob(\cat C)$, there exists an \newterm{identity morphism} $\id_X$ or $1_X$ such that if $f \in \Hom(X, Y)$, there $f \circ \id_X = f$ and if $g \in \Hom(Z, X)$, then $\id_X \circ g = g$.
\item If $F \in \Hom(X, Y)$, $g \in \Hom(Y, Z)$, and $h \in \Hom(Z, W)$, then
\begin{equation*}
    h \circ (g \circ f) = (h \circ g) \circ f.
\end{equation*}
This is the \newterm{law of associativity}.
\end{enumerate}

The collection of all morphisms of $\cat C$ is denoted by $\Mor(\cat C)$. If $f \in \Hom(X, Y)$, then we write $f \colon X \to Y$ or $\xymatrix{X \ar[r]^f & Y}$ and say that $f$ is a \newterm{morphism from $X$ to $Y$}, or that $X = \dom f$ is the \newterm{domain} and $Y = \cod f$ is the codomain of $f$.

\begin{Example}\label{ex:Categories}
\begin{enumerate}
\item\label{it:Categories2IsoObj} $\cat C = \{X, Y\}$, $\Mor(\cat C) = \{1_X, 1_Y, f \colon X \to Y, g \colon Y \to X \}$ with $1_X$ and $1_Y$ as identity morphisms and $f \circ g = 1_Y$, $g \circ f = 1_X$. Graphically,
\begin{equation*}
\boxed{\xymatrix{
X \ar @(ld, lu)^{1_X} \ar @/^/[r]^f & Y \ar @(ru, rd)^{1_Y} \ar @/^/[l]^g
}}
\end{equation*}

\item\label{it:Categories1ObjDisc} $\cat C = \{\bullet\}$, $\Mor(\cat C) = \{1\}$.

\item\label{it:CategoriesRing} $\cat{Ring}$ is the category of unital rings and unital ring homomorphisms.

\item\label{it:CategoriesMon} $\cat{Mon}$ is the category of monoids and monoid homomorphisms.

\item\label{it:CategoriesTop} $\cat{Top}$ is the category of topological spaces and continuous maps.

\item\label{it:CategoriesAbGrp} $\cat{AbGrp}$ is the category of Abelian groups and group homomorphisms.
\end{enumerate}
\end{Example}

\begin{Exercise}
\begin{enumerate}
\item In the category $\cat C$ given below, determine the compositions of all the non-identity morphisms.
\begin{equation*}
\boxed{\xymatrix{
	X \ar @(l, u)^{1_X} \ar @<+0.5ex>[rr]^f \ar @<-0.5ex>[rr]_g  \ar [dr]_k && Y \ar @(u, r)^{1_Y}  \ar[ld]^h \\
	& Z \ar @(dr, dl)^{1_Z} 
}}
\end{equation*}

\item In the category $\cat C$ given below, determine the compositions of all the non-identity morphisms.
\begin{equation*}
\boxed{\xymatrix{
	X \ar @(l, u)^{1_X} \ar @/^/[dr]^h && Y \ar @(u, r)^{1_Y} \ar[ll]_f \ar[ld]^g \\
	& Z \ar @(dr, dl)^{1_Z} \ar @/^/[ul]^k
}}
\end{equation*}

\item If $\cat C$ is a category with $\Ob(\cat C) = \{ \bullet \}$ and $\Mor(\cat C) = \{1, f, g\}$, such that $f \circ g = 1$. Then determine $f \circ f$, $g \circ g$, and $g \circ f$.

\item Show that the following is not a valid category.
\begin{equation*}
\boxed{\xymatrix{
	X \ar @(ld, lu)^{1_X} \ar  @/^1pc/[rr]^f  \ar  @/_1pc/[rr]_h && Y \ar @(ru, rd)^{1_Y}  \ar[ll]|-g
}}
\end{equation*}

\item Let $X$ be an object of a category $\cat C$. Denote by $\cat C / X$ the category whose objects are
\begin{equation*}
\Ob(\cat C / X) = \qty{\, f \colon Y \to X \mid Y \in \Ob(\cat C) \,}
\end{equation*}
morphisms are
\begin{equation*}
\Hom(f, g) = \qty{\, h \colon Y \to Z \mid g \circ h = f \,}
\end{equation*}
for all $Y, Z \in \Ob(\cat C)$, $f \colon Y \to X$, $g \colon Z \to X$, and the composition of $m \in \Hom(f, g)$ and $n \in \Hom(g, h)$ is defined to be the morphism $n \circ m$ of $\cat C$ . Show that $\cat C / X$ is indeed a category.

\item Define a category $X / \cat C$ analogous to $\cat C / X$ and show that it is a category.

\item Let $\cat C$ be a category. Define a new category $\cat C\opp$ whose objects are the same as those of $\cat C$, with a morphism $f\opp \colon Y \to X$ corresponding to every morphism $f \colon X \to Y$ in $\cat C$, and composition defined by
\begin{equation*}
f\opp \circ g\opp = (g \circ f)\opp
\end{equation*}
for all $f\opp \colon Y \to X$, $g\opp \colon Z \to Y$. Show that $\cat C\opp$ is indeed a category (see \cref{sec:Duality}).

\item What is the category $(\cat C / X)\opp$?

\item Can the following be a category? If so, how?
\begin{equation*}
\boxed{\xymatrix{
	& \bullet \ar @(ru, lu)_f \ar @(ld, rd)_1 &
}
}
\end{equation*}

\item Can the following be a category? If so, how?
\begin{equation*}
\boxed{\xymatrix{
	X \ar @(ld, lu)^{1_X} \ar  @/^1pc/[rr]^f  \ar  @/_1pc/[rr]_h && Y \ar @(ul, ur)^k \ar @(ur, dr)^{1_Y} \ar[ll]|-g
}}
\end{equation*}

\item Can the following be a category? If so, how?
\begin{equation*}
\boxed{\xymatrix{
	X \ar @(ld, lu)^{1_X} \ar  @/^1pc/[rr]^f  \ar  @/_1pc/[rr]_h && Y \ar @(ul, ur)^k \ar @(ur, dr)^{1_Y} \ar @(dr, dl)^l \ar[ll]|-g
}}
\end{equation*}
\end{enumerate}
\end{Exercise}

A category is \newterm{small} if its collections of objects and morphisms are sets -- note that this is equivalent to simply saying that its collection of morphisms, as the objects are in one-to-one correspondence with the identity morphisms. A category that is not small is \newterm{large}. A category is \newterm{locally small} if $\Hom(X, Y)$ is a set for every two objects $X$ and $Y$ of the category. In this case, $\Hom(X, Y)$ is said to be a \newterm{hom-set}.

In \cref{ex:Categories}, the categories given in \labelcref{it:Categories2IsoObj,it:Categories1ObjDisc} are small (and hence also locally small), whereas the ones given in \labelcref{it:CategoriesRing,it:CategoriesMon,it:CategoriesAbGrp} are large but locally small.

\section{Duality}\label{sec:Duality}
The \newterm{opposite category} of a category $\cat C$ is the category $\cat C\opp$ with $\Ob(\cat C\opp) = \Ob(\cat C)$ and for each pair of objects $X$ and $Y$, \spliteq$\Hom_{\cat C\opp}(X, Y) = \qty{\, f\opp \mid f \in \Hom_{\cat C}(Y, X) \,}$, in which composition is defined as
\begin{equation*}
g\opp \circ f\opp = (f \circ g)\opp
\end{equation*}
for all $f\opp \in \Hom_{\cat C\opp}(X, Y)$ and $g\opp \in \Hom_{\cat C\opp}(Y, Z)$, for all objects $X$, $Y$, and $Z$.

The fact that $\cat C\opp$ is a category gives rise to the principle of duality -- if a statement is true in all categories, then its \newterm{dual statement}, obtained by reversing the directions of all morphisms, must also be true in all categories. In fact, every concept defined in category theory has a dual concept. For instance, the concepts of initial and final objects are duals of each other. We will later prove that any two initial objects in a category must be isomorphic to each other. From this, it immediately follows, by the principle of duality, that any two final objects in a category must be isomorphic to each other.


\section{Diagram Chasing}\label{sec:DiagramChasing}

A diagram of objects and morphisms \newterm{commutes} (or is \newterm{commutative}) if, whenever there are two directed paths with the same source and same target, the composition of the morphisms along one path equals the composition of the morphisms along the other. That is, for each pair of paths
\begin{equation*}
\xymatrix{
& \bullet \ar[r]^{f_2} &  \bullet & \cdots & \bullet \ar[r]^{f_{m-1}} & \bullet \ar[rd]^{f_m} & \\
\bullet \ar[ru]^{f_1} \ar[rd]_{g_1} &&&&&& \bullet \\
& \bullet \ar[r]_{g_2} & \bullet & \cdots & \bullet \ar[r]_{g_{n-1}} & \bullet \ar[ru]_{g_n} &
}
\end{equation*}
in the diagram, $f_m \circ f_{m-1} \circ \cdots \circ f_2 \circ f_1 = g_n \circ g_{n-1} \circ \cdots \circ g_2 \circ g_1$.

\begin{Example} In the diagram below, suppose $h = g \circ f$.
\begin{equation*}
\xymatrix{
X \ar[r]^f \ar[d]_h & Y \ar[d]^{1_Y} \\
Z & Y \ar[l]^g
}
\end{equation*}
In this diagram, there are two different directed paths from $X$ to $Z$, one consisting of the morphisms $f$, $1_Y$, and $g$, and the other consisting of a single morphism $h$. The composition of the morphisms along the first path is $g \circ 1_Y \circ f = g \circ f = h$, and the composition of the morphisms along the second path is simply $h$. As both of these are equal, the diagram commutes.
\end{Example}

\begin{Example}
Consider $f \colon \mathbb C \to \mathbb C$, $f(z) = -iz$, $g \colon \mathbb C \to \mathbb R$, $g(z) = \Re(z)$, and $h \colon \mathbb C \to \mathbb R$, $h(z) = \Im(z)$, for all $z \in \mathbb C$. Then the following diagram commutes:
\begin{equation*}
\xymatrix{
\mathbb C \ar[r]^f \ar[rd]_h & \mathbb C \ar[d]^g \\
& \mathbb C
}
\end{equation*}
To verify this, we could take an arbitrary element of $\mathbb C$ and ``chase'' it across the diagram to see if it maps to the same element along all paths that lead to the same target. We may represent this visually as given below.
\begin{equation*}
\xymatrix@C=3.2em{
x + iy \ar@{|->}[r]^{(-i) \times \leftrightline} \ar@{|->}[rd]_{\Im} & y - ix \ar@{|->}[d]^{\Re} \\
& y
}
\end{equation*}
\end{Example}

\begin{Exercise}
Let $f \colon \mathbb C \to \mathbb C$, $f(z) = \overline z$, $g \colon \mathbb C \to \mathbb C$, $g(z) = z^2$, and $h \colon \mathbb C \to \mathbb C$, $h(z) = \Re(z)$. Which of the following is/are commutative?
\begin{figure}[H]
\centering
\subcaptionbox{}{
	$\xymatrix{
	 	\mathbb C \ar[r]^f \ar[d]_g & \mathbb C \ar[d]^g \\
 	 	\mathbb C \ar[r]_f & \mathbb C 
 	}$
} \qquad
\subcaptionbox{}{
	$\xymatrix{
		\mathbb C \ar[r]^f \ar[d]_h & \mathbb C \ar[d]^{h} \\
		\mathbb C \ar[r]_f & \mathbb C 
	}$
}
\end{figure}
\end{Exercise}

\begin{Example}
The law of associativity of composition can be stated in terms of a commutative diagrams as follows. For all morphisms $f$, $g$, $h$ with domains and codomains as shown in the diagram below, if each subdiagram with three vertices commutes, then the whole diagram commutes.
\begin{equation*}
\xymatrix{
\bullet \ar[r]^f \ar@/^2pc/[rr] \ar@/^3pc/[rrr]^{} \ar@/_3pc/[rrr]_{}
& \bullet \ar[r]^g \ar@/_2pc/[rr]
& \bullet \ar[r]^h & \bullet
}
\end{equation*}

The identity morphism of an object $X$ can be defined as the morphism such that all diagrams of the following form commute.
\begin{equation*}
\xymatrix@=3em{
X \ar[r]^f \ar[rd]^{1_X} \ar[d]_g & Y \ar[d]^f \\
Z \ar[r]_g & X
}
\end{equation*}
\end{Example}

Very often, the commutativity of a larger diagram can be checked by verifying the commutativity of its smaller components. Commutative diagrams of the forms given below are called \newterm{commuting triangles} and \newterm{commuting squares}, respectively.
\begin{align*}
\xymatrix{
\bullet \ar[r] \ar[rd] & \bullet \ar[d]\\
& \bullet
}
&&
\xymatrix{
\bullet \ar[r] \ar[d] & \bullet \ar[d] \\
\bullet \ar[r] & \bullet
}
\end{align*}

\begin{Exercise}[{\cite[Exercise 1.2]{SchalkSimmons2005}}]
In each of the diagrams given below, show that if the two triangles commute, then the square commutes.
\begin{figure}[H]
\centering
\subcaptionbox{\label{subfig:CommSq1}}{
$\xymatrix@=3em{
\bullet \ar[r] \ar[d] \ar[rd] & \bullet \ar[d] \\
\bullet \ar[r] & \bullet
}$
}
\hspace{0.3\textwidth}
\subcaptionbox{\label{subfig:CommSq2}}{
$\xymatrix@=3em{
\bullet \ar[r] \ar[d] & \bullet \ar[d] \ar[ld]\\
\bullet \ar[r] & \bullet
}$
}
\end{figure}
\end{Exercise}

\begin{Solution*}
In \subref{subfig:CommSq1}, from the commutativity of the upper triangle, the composition of the top and right sides of the square equals the diagonal, which, from the commutativity of the lower triangle, equals the composition of the left and bottom sides. This shows that the square commutes.

In \subref{subfig:CommSq2}, since the lower triangle commutes, the right side equals the composition of the diagonal and the bottom side. Hence, the composition of the top and right sides equals the composition of the top side, the diagonal, and the bottom side. Similarly, from the commutativity of the upper triangle, the composition of the top side, the diagonal, and the bottom side also equals the composition of the bottom and left sides. Thus, the square commutes.
\end{Solution*}

\note{Arguments such as the above one are much easier to \emph{see} than to understand by reading, but this requires some training. The writing itself could be made considerably simpler by naming the arrows (the morphisms) in the diagram and expressing the argument in the form of equations, but the aim of these exercises is to develop the skill of ``diagram chasing''. Attempt the remaining exercises in the same manner, avoiding the urge to name arrows, as much as possible.}

\begin{Exercise}[{\cite[Exercise 1.2]{SchalkSimmons2005}}]
In the diagram given below, viewed as a tetrahedron, show that if the left, right, and bottom faces commute, then the back face also commutes.
\begin{equation*}
\xymatrix@=3em{
& \bullet \ar[ld] \ar[dd] \ar[rd] & \\
\bullet \ar[rd] \ar[rr]|\hole & & \bullet \\
& \bullet \ar[ru] &
}
\end{equation*}
\end{Exercise}

\begin{Exercise}
In each of the diagrams below, show that if both the squares commute, then the outer rectangle commutes.
\begin{figure}[H]
\centering
\subcaptionbox{}{$\xymatrix@=3em{
\bullet \ar[r] \ar[d] & \bullet \ar[r] \ar[d] & \bullet \ar[d] \\
\bullet \ar[r] & \bullet \ar[r] & \bullet
}$}
\hspace{0.1\textwidth}
\subcaptionbox{}{$\xymatrix@=3em{
\bullet \ar[r] \ar[d] & \bullet \ar[r] \ar[d] & \bullet \\
\bullet \ar[r] & \bullet \ar[r] & \bullet \ar[u]
}$}
\hspace{0.1\textwidth}
\subcaptionbox{}{$\xymatrix@=3em{
\bullet \ar[r] \ar[d] & \bullet \ar[r] & \bullet \ar[d] \\
\bullet \ar[r] & \bullet \ar[r] \ar[u] & \bullet
}$}
\end{figure}
\end{Exercise}

\begin{Exercise}
Show that if the inner triangle and the three trapeziums in the diagram below commute, then the outer triangle commutes.
\begin{equation*}
\xymatrix{
&&& \bullet \ar[d] \ar[rrrddd] &&& \\
&&& \bullet \ar[rd] &&& \\
&& \bullet \ar[ur] \ar[rr] && \bullet \ar[rrd] && \\
\bullet \ar[rrruuu] \ar[rru] \ar[rrrrrr] &&&&&& \bullet
}
\end{equation*}
\end{Exercise}

\begin{Exercise}[{\cite[Exercise 1.4]{SchalkSimmons2005}}]
Show that if the inner square and the four trapeziums in the diagram below commute, then the outer square commutes.
\begin{equation*}
\xymatrix@=2.5em{
\bullet \ar[ddd] \ar[rd] \ar[rrr] &&& \bullet \ar[ld] \ar[ddd] \\
& \bullet \ar[d] \ar[r] & \bullet \ar[d] & \\
& \bullet \ar[r] & \bullet \ar[rd] & \\
\bullet \ar[ru] \ar[rrr] &&& \bullet
}
\end{equation*}
\end{Exercise}

\begin{Exercise}
Let $G$ and $H$ be groups. If $f \colon G \to H$ is any function, define $f \times f \colon G \times G \to H \times H$ by $f(x, y) = (f(x), f(y))$, for all $x, y \in G$. Let $\cdot$ and $*$ denote the group operations in $G$ and $H$, respectively. Then show that $f$ is a homomorphism if and only if the following diagram commutes.
\begin{equation*}
\xymatrix@C=5em@R=4em{
G \times G \ar[r]^{f \times f} \ar[d]_\cdot & H \times H \ar[d]^{*} \\
G \ar[r]_f & H
}
\end{equation*}
\end{Exercise}

\begin{Exercise}
Show that in the diagram below, if all the parallelograms commute, and any one of the two squares commutes, then all directed paths with three arrows are equal under composition.
\begin{equation*}
\xymatrix@=3em{
& \bullet \ar[r] \ar[rd] & \bullet \ar[rd] & \\
\bullet \ar[r] \ar[rd] \ar[ru] & \bullet \ar[rd] \ar[ru] & \bullet \ar[r] & \bullet \\
& \bullet \ar[r] \ar[ru] & \bullet \ar[ru] &
}
\end{equation*}
\end{Exercise}

\section{Some Classes of Morphisms}\label{sec:MorphismClasses}

A morphism $f \colon X \to Y$ in a category $\cat C$ is an \newterm{isomorphism} if there exists a morphism $g \colon Y \to X$ in $\cat C$ such that $g \circ f = 1_X$ and $f \circ g = 1_Y$. Such a morphism $g$ is an \newterm{inverse} of $f$. If there exists an isomorphism $f \colon X \to Y$, then $X$ is \newterm{isomorphic} to $Y$, written $X \cong Y$.

Equivalently, we can define an isomorphism as a morphism $f$ such that there exists a morphism $g$ making the following diagram commute.
\begin{equation*}
\xymatrix@C=2.5em{
Y \ar[r]^g \ar@/^2.2pc/[rr]^{1_Y} & X \ar[r]^f \ar@/_2.2pc/[rr]_{1_X} & Y \ar[r]^g & X
}
\end{equation*}

\begin{Exercise}
\begin{enumerate}
\item Show that each identity morphism is an isomorphism.
\item Show that if $f$ is an isomorphism, then it has a unique inverse, and the inverse is also an isomorphism.
\item Show that the composition of two compatible isomorphisms is also an isomorphism.
\item Show that the isomorphism relation $\cong$ is an equivalence relation on the collection of objects of a category.
\end{enumerate}
\end{Exercise}

\begin{Example}
\begin{enumerate}
\item In $\cat{Set}$, the isomorphisms are exactly the bijective functions.
\item In $\cat{Grp}$, $\cat{Mon}$, $\cat{Ring}$, and $\cat{AbGrp}$, the isomorphisms are the bijective homomorphisms of the respective algebraic structures.
\item In $\cat{Vect}_F$, the isomorphisms are the bijective linear transformations.
\item In $\cat{Top}$, the isomorphisms are the homeomorphisms. Note that a homeomorphism is not simply a bijective continuous map (i.e.\ a morphism of $\cat{Top}$ that is bijective as a function). Why?
\item In the category given in \cref{ex:Categories}, \labelcref{it:Categories2IsoObj}, all the morphisms are isomorphisms.
\end{enumerate}
\end{Example}

We may weaken the definition of isomorphism in two ways as follows. If $r \colon X \to Y$ and $s \colon Y \to X$ are morphisms such that $r \circ s = 1_Y$, then $r$ is a \newterm{retraction} and $s$ is a \newterm{section}.
\begin{equation*}
\xymatrix{
Y \ar[r]^s \ar[rd]_{1_Y} & X \ar[d]^r \\
& Y
}
\end{equation*}
We also say that $r$ is a \newterm{left inverse} of $s$ and that $s$ is a \newterm{right inverse} of $r$. Thus, a retraction is a right-invertible morphism and a section is a left-invertible morphism.

\begin{Exercise}
Prove that a morphism is an isomorphism if and only if it is both a retraction and a section.

\note{The statement that a morphism is both a retraction and a section only tells you that it has some right inverse and some left inverse -- not that the right and left inverses are the same. That is, if $f \colon X \to Y$ is both a retraction and a section, you may only assume that there exist morphisms $s \colon Y \to X$ and $r \colon Y \to X$ such that $f \circ s = 1_Y$ and $r \circ f = 1_X$.}

\hint{Show that $r = s$.}
\end{Exercise}

\begin{Example}\label{ex:SetRetSec}
In the category $\cat{Set}$, consider the sets $X = \mathbb R$ and $Y = \mathbb Z$. Define $r \colon X \to Y$ by $r(x) = \floor x$ (the greatest integer not exceeding $x$) for all $x \in X$, and $s \colon Y \to X$ by $s(y) = y$ for all $y \in Y$. Then $r \circ s = \id_Y$. Note that neither $r$, nor $s$ is an isomorphism.
\end{Example}

\begin{Exercise}
\begin{enumerate}
\item Show that $f$ is a retraction in a category $\cat C$ if and only if $f\opp$ is a section in $\cat C\opp$.
\item Show that in $\cat{Set}$, every section is an injective function and every retraction is a surjective function.
\item Show that in $\cat{Set}$, every injective function is a section.
\item Read about the Axiom of Choice and try to understand its equivalence to the following statement: In $\cat{Set}$, every surjective function is a retraction.
\item Show that in $\cat{Grp}$, every section is an injective group homomorphism.
\item In $\cat{Grp}$, consider the inclusion map $\iota \colon \mathbb Z \to \mathbb R$ defined by $\iota(n) = n$ for all $n \in \mathbb Z$ (where $\mathbb Z$ and $\mathbb R$ are the additive groups defined as usual). Show that $\iota$ is an injective homomorphism, but is not a section.
\end{enumerate}
\end{Exercise}

The definitions of sections and retractions can be further weakened in the following manner. A \newterm{monomorphism} is a morphism $m \colon X \to Y$ such that for all $Z$ and $f, g \colon Z \to X$, if $m \circ f = m \circ g$, then $f = g$. In other words, $m$ is \newterm{left-cancellable}. We also say that $m$ is \newterm{monic}. Dually, an \newterm{epimorphism} is a morphism $e \colon X \to Y$ such that for all $Z$ and $f, g \colon Y \to Z$, if $f \circ e = g \circ e$, then $f = g$. That is, $e$ is \newterm{right-cancellable}. We also say that $e$ is \newterm{epic}.

\begin{Exercise}
Show that monomorphisms and epimorphisms are indeed dual concepts.
\end{Exercise}

Observe that the defining property of a monomorphism, namely $m \circ f = m \circ g \implies f = g$, appears similar to that of an injective functions between sets, namely $m(x) = m(y) \implies x = y$ (for elements $x$ and $y$ of the domain). Indeed, monomorphisms are the categorical generalisations of injective functions. This idea is formalised in the statement of the next exercise.

\begin{Exercise}
Show that monomorphisms in the category $\cat{Set}$ are exactly injective functions.
\end{Exercise}

\begin{Solution*}
Consider a function $m \colon X \to Y$ in $\cat{Set}$. First, suppose that $m$ is a monomorphism. That is, if $f$ and $g$ are functions from some set $Z$ to $X$ such that $m \circ f = m \circ g$, then $f = g$. We need to show that $m$ is an injection. That is, if $x$ and $y$ are two elements of $X$ such that $m(x) = m(y)$, then $x = y$. Let $Z$ be a singleton set, say $Z = \{\bullet\}$. Define functions $f, g \colon Z \to X$ by $f(\bullet) = x$ and $g(\bullet) = y$.
\begin{center}
\includegraphics[scale=1.2]{Mono=Injection.pdf}
\end{center}
Then, $m(f(\bullet)) = m(x)$, and $m(g(\bullet)) = m(y)$. Now, suppose that $m(x) = m(y)$. Then we have $m(f(\bullet)) = m(g(\bullet))$, which implies that $m \circ f = m \circ g$ (since $\bullet$ is the only element of the domain $Z$ of these two functions). But $m$ is a monomorphism, and therefore we have $f = g$, which implies $f(\bullet) = g(\bullet)$, i.e. $x = y$, as required.

Next, suppose that $m$ is an injection. If $f, g \colon Z \to X$ are such that $m \circ f = m \circ g$, then for all $z \in Z$, $m(f(z)) = m(g(z))$. As $m$ is injective, this implies that $f(z) = g(z)$, for all $z \in Z$, i.e. $f = g$, showing that $m$ is a monomorphism.
\end{Solution*}

Similarly, epimorphisms are the categorical generalisations of surjective functions, as you are asked to show in the next exercise.

\begin{Exercise}
Show that epimorphisms in the category $\cat{Set}$ are exactly surjective functions.
\end{Exercise}

\begin{Exercise}
Let $f \colon X \to Y$ and $g \colon Y \to Z$ be morphisms in a category. Prove the following:
\begin{enumerate}
\item If $f$ and $g$ are monic, then so is $g \circ f$.
\item If $f$ and $g$ are epic, then so is $g \circ f$.
\item If $g \circ f$ is monic, then so is $f$.
\item If $g \circ f$ is epic, then so is $g$.
\end{enumerate}
\end{Exercise}

\begin{Exercise}
Prove that every monic retraction is an isomorphism and that every epic section is an isomorphism.
\end{Exercise}

\begin{Exercise}[{\cite[Exercise 1.4]{SchalkSimmons2005}}]
Show that in the diagram given below, if $e$ is epic, $m$, is monic, and the outer square and the four trapeziums commute, then the inner square commutes.
\begin{equation*}
\xymatrix@=2.5em{
\bullet \ar[ddd] \ar[rd]^e \ar[rrr] &&& \bullet \ar[ld] \ar[ddd] \\
& \bullet \ar[d] \ar[r] & \bullet \ar[d] & \\
& \bullet \ar[r] & \bullet \ar[rd]_m & \\
\bullet \ar[ru] \ar[rrr] &&& \bullet
}
\end{equation*}
\end{Exercise}

\begin{Exercise}[{\cite[Exercise 1.3]{SchalkSimmons2005}}]
Consider the cube given below.
\begin{equation*}
\xymatrix@=2.5em{
\bullet \ar[rr] && \bullet & \\
& \bullet \ar[lu] \ar[rr] && \bullet \ar[lu] \\
\bullet \ar[uu] \ar[rr]|\hole && \bullet \ar[uu]|\hole^(0.3)m & \\
& \bullet \ar[lu]_e \ar[uu] \ar[rr] && \bullet \ar[lu] \ar[uu]
}
\end{equation*}
\begin{enumerate}
\item Show that if $e$ is epic and if the left, top, front, right, and bottom faces commute, then the back face also commutes.
\item Show that if $m$ is monic and if the back, right, front, top, and left faces commute, then the bottom face also commutes.
\end{enumerate}
\end{Exercise}


\section{Products and Coproducts}\label{sec:ProductsCoproducts}

Familiar mathematical structures such as groups, rings, vector spaces, and topological spaces have some notion of a product structure. In all these cases, where the mathematical structures themselves are sets with some additional structure, the product is defined as Cartesian product of the underlying sets, together with some additional structure. Recall that the Cartesian product of two sets $A$ and $B$ is the set $A \times B = \qty{\, (a, b) \mid a \in A, b \in B \,}$. Such a definition cannot be given in an arbitrary category, since an object of the category may not have any ``underlying set''. Instead, we define products (and their duals, coproducts) in a category by using \newterm{universal properties}.

The \newterm{product} of two objects $X$ and $Y$ of a category $\cat C$ is an object $P$, together with morphisms $\pi_X \colon P \to X$ and $\pi_Y \colon P \to Y$, such that for every object $Z$ and morphisms $f \colon Z \to X$ And $g \colon Z \to Y$, there exists a unique morphism $h \colon Z \to P$ that makes the following diagram commute:
\begin{equation*}
\xymatrix@=3em@C=5em{
& X \\
Z \ar[ru]^f \ar[rd]_g \ar@{-->}[r]^h & P \ar[u]_{\pi_X} \ar[d]^{\pi_Y}\\
& Y
}
\end{equation*}
If such a triple $(P, \pi_A, \pi_B)$ does not exist, then $A$ and $B$ have no product in $\cat C$.

\note{The product is not merely the object $P$, but the triple $(P, \pi_X, \pi_Y)$. In particular categories, these maps may be obvious, and therefore we may usually refer to only $P$ as the product. But in an arbitrary category, the actual properties of the product are carried by these two morphisms rather than by the product object itself.}

\begin{Example}\label{ex:CartesianProduct}
First, let us see how this definition specialises to the Cartesian product in the category $\cat{Set}$. For any two sets $A$ and $B$, there are projection maps $\pi_A \colon A \times B \to A$ and $\pi_B \colon A \times B \to B$ from the Cartesian product $A \times B$ to the sets $A$ and $B$ respectively, defined by $\pi_A(a, b) = a$ and $\pi_B(a, b) = b$ for all $(a, b) \in A \times B$. We shall therefore verify that the triple $(A \times B, \pi_A, \pi_B)$ satisfies the universal property of the product.

Let $C$ be any set and $f \colon C \to A$ and $g \colon C \to B$ two functions from $C$ to $A$ and $B$, respectively. We must show the existence of a unique function $h \colon C \to A \times B$ such that
\begin{equation*}
\xymatrix@=3em@C=5em{
& A \\
C \ar[ru]^f \ar[rd]_g \ar@{-->}[r]^h & A \times B \ar[u]_{\pi_A} \ar[d]^{\pi_B}\\
& B
}
\end{equation*}
commutes -- i.e. such that $\pi_A \circ h = f$ and $\pi_B \circ h = g$.

Define $h$ by $h(c) = (f(c), g(c))$, for all $c \in C$. Note that indeed, $h(c) \in A \times B$, since $f(c) \in A$ and $g(c) \in B$.

Next, observe that $\pi_A(h(c)) = f(c)$, and $\pi_B(h(c)) = g(c)$, which shows that $\pi_A \circ h = f$ and $\pi_B \circ h = g$, as required.

Finally, to see that $h$ is the unique map satisfying these conditions, suppose that $h(c) = (a, b) \in A \times B$. Then from $\pi_A \circ h = f$, we have $\pi_A(a, b) = f(c)$, i.e. $a = f(c)$, and similarly $b = g(c)$. Therefore, $h(c) = (f(c), g(c))$ necessarily.
\end{Example}

\begin{Exercise}
Verify that the triple $(B \times A, \pi_B, \pi_A)$ (with notation as given in \cref{ex:CartesianProduct}) also satisfies the definition of the product of $A$ and $B$. Conclude that the product of two objects need not be unique. Are $A \times B$ and $B \times A$ isomorphic to each other? (Recall that an isomorphism in $\cat{Set}$ is merely a bijection).
\end{Exercise}

\begin{Example}
Let $G$ and $H$ be two groups. The direct product of $G$ and $H$ is the group defined by the Cartesian product $G \times H$ together with componentwise operations -- i.e. the group operation is defined as $(x_1, y_1)(x_2, y_2) = (x_1 x_2, y_1, y_2)$ for all $(x_1, y_1), (x_2, y_2) \in G \times H$.

As the underlying set of the direct product is the Cartesian product $G \times H$, the projection maps $\pi_G$ and $\pi_H$ from $G \times H$ to $G$ and $H$ are naturally defined. But in order for $(G \times H, \pi_G, \pi_H)$ to be the product of $G$ and $H$ in the category $\cat{Grp}$, the maps $\pi_G$ and $\pi_H$ must be morphisms in $\cat{Grp}$, i.e. they must be group homomorphisms. It is easily verfied that they are.

Again, it is clear that if $K$ is any group and $p \colon K \to G$ and $q \colon K \to H$ are group homomorphisms, then there is a unique group homomorphism $m \colon K \to G \times H$ defined by $m(k) = (p(k), q(k))$ for all $k \in K$, such that $\pi_G \circ m = p$ and $\pi_H \circ m = q$.
\end{Example}

The \newterm{coproduct} of two objects $X$ and $Y$ of a category $\cat C$ is an object $S$, together with morphisms $\iota_X \colon X \to S$ and $\iota_Y \colon Y \to S$, such that for every object $Z$ and morphisms $f \colon X \to Z$ And $g \colon Y \to Z$, there exists a unique morphism $h \colon S \to Z$ that makes the following diagram commute:
\begin{equation*}
\xymatrix@=3em@C=5em{
X \ar[d]_{\iota_X} \ar[rd]^{f} \\
S \ar@{-->}[r]^h & Z\\
Y \ar[u]^{\iota_Y} \ar[ru]_{g}
}
\end{equation*}
If no such triple $(S, \iota_X, \iota_Y)$ exists, then $X$ and $Y$ have no coproduct in $\cat C$.

\begin{Example}\label{ex:DisjointUnion}
Consider two sets $A$ and $B$. Their coproduct in the category $\cat{Set}$ is the \newterm{disjoint union}
\begin{equation*}
A \sqcup B = A \times \{1\} \cup B \times \{2\} = \qty{\, (a, 1) \mid a \in A \,} \cup \qty{\, (b, 2) \mid b \in B \,}.
\end{equation*}
The injections $\iota_A \colon A \to A \cup B$ and $\iota_B \colon B \to A \cup B$ are given by $\iota_A(a) = (a, 1)$ for all $a \in A$, and $\iota_B(b) = (b, 2)$ for all $b \in B$.
\end{Example}

\note{Intuitively, the disjoint union of two sets is simply their union considering the two sets to be disjoint. That is, we relabel an element $a$ from $A$ as $(a, 1)$ to distinguish it, as an element of $A \sqcup B$, from an element coming from $B$, which would be labelled as $(b, 2)$. For example, if $A = \{x, y, z\}$ and $B = \{y, w\}$, then $A \sqcup B = \{(x, 1), (y, 1), (z, 1), (y, 2), (w, 2)\}$. Even though $y$ is common to both $A$ and $B$, in the disjoint union we have two ``copies'' of $y$, labelled as $(y, 1)$ and $(y, 2)$ to identify where each copy comes from.}

\begin{Exercise}
Verify that the disjoint union of two sets as defined in \cref{ex:DisjointUnion} is indeed their coproduct.
\end{Exercise}

\begin{Exercise}
Let $A$ and $B$ be two Abelian groups, considered as objects in the category $\cat{AbGrp}$ of Abelian groups and group homomorphisms.
\begin{enumerate}
\item Verify that the product of $A$ and $B$ is their direct product $A \times B$, as defined in the category $\cat{Grp}$ itself.

\item Define $\iota_A \colon A \to A \times B$ by $\iota_A(a) = (a, e_B)$ for all $a \in A$ (where $e_B$) denotes the identity element of $B$). Verify that $\iota_A$ is a group homomorphism. Similarly define $\iota_B \colon B \to A \times B$ and observe that it is a group homomorphism.

\item Suppose that $C$ is an Abelian group and $f \colon A \to C$ and $g \colon B \to C$ are group homomorphisms. Show that there exists a group homomorphism $h \colon A \times B \to C$ such that $h \circ \iota_A = f$ and $h \circ \iota_B = g$.

\hint{Define $h$ as $h(a,b) = f(a)g(b)$.}

\item Verify that $h$ is the unique such homomorphism.

\item Conclude that $A \times B$ is the coproduct of $A$ and $B$ in $\cat{AbGrp}$.
\end{enumerate}
\end{Exercise}

\section{Subcategories}

As with most mathematical structures, categories also have substructures. A \newterm{subcategory} of a category $\cat C$ is a category $\cat D$ such that $\Ob(\cat D) \subseteq \Ob(\cat C)$ and $\Mor(\cat D) \subseteq \Mor(\cat C)$. The latter is equivalent to the statement that if $X$ and $Y$ are objects of $\cat D$, then $\Hom_{\cat D}(X, Y) \subseteq \Hom_{\cat C}(X, Y)$. Clearly, every category is a subcategory of itself, and the empty category, which has no objects and no morphisms, is a subcategory of every category.

A subcategory $\cat D$ of a category $\cat C$ is \newterm{wide} if all objects of $\cat C$ are also present in $\cat D$. Similarly, $\cat D$ is a \newterm{full} subcategory of $\cat C$ if it is a subcategory of $\cat C$ such that for all objects $X, Y \in \Ob(\cat D)$, $\Hom_{\cat D}(X, Y) = \Hom_{\cat C}(X, Y)$.

\begin{Example}
\begin{enumerate}
\item $\cat{Set}$ is a subcategory of $\cat{Rel}$ -- both share the same objects, and every morphism in $\cat{Set}$ (a function) is a morphism in $\cat{Rel}$ (a relation). Hence, it $\cat{Set}$ is a wide subcategory of $\cat{Rel}$, but not a full subcategory, since not every relation is a function.
\item $\cat{AbGrp}$ is a subcategory of $\cat{Grp}$, which is itself a subcategory of $\cat{Mon}$. In each case, the subcategory is full, but not wide.
\item $\cat{Mon}$ is a subcategory of $\cat{Semigrp}$, the category of semigroups and semigroup homomorphisms. Note that it is neither a wide subcategory (obvious), nor a full subcategory (why?).
\item For any category $\cat C$, the collection $\cat C^\mathsf{epi}$ consisting of all objects of $\cat C$ and all (and only) the epimorphisms of $\cat C$ forms a (wide) subcategory of $\cat C$. The subcategories $\cat C^\mathsf{mono}$ and $\cat C^\mathsf{iso}$ can be defined similarly.
\item For any category $\cat C$, the collection of all objects of $\cat C$ together with only the identity morphisms is a wide subcategory of $\cat C$.
\item For any object $X$ of a category $\cat C$, the collection consisting of just $X$ and all morphisms from $X$ to itself is a full subcategory of $\cat C$.
\end{enumerate}
\end{Example}

\section{Functors}

Categories themselves are mathematical structures, and so we are naturally led to consider categories whose objects are themselves categories\footnote{There are set-theoretic problems in attempting to define a \emph{category of all categories}, but this is easily overcome, for instance, by only considering the category of all \emph{small} categories.}. What should be the morphisms in such a category? Just as the structure of an algebraic object (such as a group or a vector space) is considered to be determined by its operation(s), the structure of a category is determined by how its morphisms are composed. A structure-preserving map between categories should therefore preserve compositions and identity morphisms.

A \newterm{(covariant) functor} from a category $\cat C$ to a category $\cat D$ is a function $F \colon \cat C \to \cat D$ satisfying the following properties:
\begin{enumerate}[label = (\roman*)]
\item $F$ maps each object $X$ of $\cat C$ to an object $FX$ of $\cat D$.
\item $F$ maps each morphism $f \colon X \to Y$ in $\cat C$ to a morphism $F(f) \colon FX \to FY$ of $\cat D$.
\item If $f \colon X \to Y$ and $g \colon Y \to Z$ are morphisms in $\cat C$, then $F(g \circ f) = F(g) \circ F(f)$.
\item For each object $X$ of $\cat C$, $F(1_X) = 1_{FX}$.
\end{enumerate}

These properties imply that $F$ preserves commutativity of diagrams -- i.e. given a commutative diagram in $\cat C$, the diagram obtained by replacing each object $X$ in the diagram by $FX$ and each morphism $f$ in the diagram by $F(f)$ is a commutative diagram in $\cat D$.

A \newterm{contravariant functor} $F$ from $\cat C$ to $\cat D$ is a covariant functor from $\cat C\opp$ to $\cat D$. To make this evident, we shall write $F \colon \cat C\opp \to D$. Thus, a contravariant functor $F \colon \cat C\opp \to D$ is a function satisfying the following properties:
\begin{enumerate}[label = (\roman*)]
\item $F$ maps each object $X$ of $\cat C$ to an object $FX$ of $\cat D$.
\item $F$ maps each morphism $f \colon X \to Y$ in $\cat C$ to a morphism $F(f) \colon FY \to FX$ of $\cat D$.
\item If $f \colon X \to Y$ and $g \colon Y \to Z$ are morphisms in $\cat C$, then $F(g \circ f) = F(f) \circ F(g)$.
\item For each object $X$ of $\cat C$, $F(1_X) = 1_{FX}$.
\end{enumerate}

Obviously, a contravariant functor also preserves commutativity of diagrams. We shall write \emph{functor} to mean a covariant functor, and a contravariant functor will always be explicitly stated to be so.

\begin{Example}
\begin{enumerate}
\item For any category $\cat C$, there is an \newterm{identity functor} $I \colon \cat C \to \cat C$ that maps each object to itself and each morphism to itself.

\item Let $\cat C$ be $\cat{Grp}$ (or $\cat{Ring}$ or $\cat{Vect}_F$ or $\cat{Top}$, etc.). Then there is a functor $U \colon \cat C \to \cat{Set}$ that maps each object $X$ to its underlying set $UX$, and any morphism $f \colon X \to Y$ to the function $f \colon UX \to UY$. This functor is said to be a \newterm{forgetful functor}, as it ``forgets'' some of the structure but otherwise leaves the object unchanged.

\item There are also forgetful functors from $\cat{AbGrp}$ to $\cat{Grp}$, from $\cat{Grp}$ to $\cat{Mon}$, and from $\cat{Mon}$ to $\cat{Semigrp}$, defined in the obvious way.

\item There is a contravariant functor $(-)^*$ from $\cat{Vect}_F$ to $\cat{Vect}_F$ that maps each vector space $V$ to its dual space $V^*$, and each linear transformation $T \colon V \to W$ to its dual transformation $T^* \colon W^* \to V^*$. Recall that $T^*$ is defined by $T^*(w^*) = (v \mapsto w^*(T(v)))$.

\item The \newterm{covariant power set functor} $\mathcal P \colon \cat{Set} \to \cat{Set}$ maps each set $X$ to its power set $\mathcal P(X)$, and each function $f \colon X \to Y$ to the function $\mathcal P(f) \colon \mathcal P(X) \to \mathcal P(Y)$ defined by
\begin{equation*}
\mathcal P(f)(A) = \qty{\, f(a) \mid a \in A \,}
\end{equation*}
for all $A \subseteq X$. In other words, $\mathcal P(f)(A)$ is the image of $A$ under $f$.

\item The \newterm{contravariant power set functor} $\overline{\mathcal P} \colon \cat{Set}\opp \to \cat{Set}$ maps each set $X$ to its power set $\overline{\mathcal P}(X) = \mathcal P(X)$ and each function $f \colon X \to Y$ to the function $\overline{\mathcal P}(Y) \to \overline{\mathcal P}(X)$ defined by
\begin{equation*}
\overline{\mathcal P}(f)(B) = \qty{\, x \in X \mid f(x) \in B \,}
\end{equation*}
for all $B \subseteq Y$. In other words, $\overline{\mathcal P}(B)$ is the preimage of $B$ under $f$.

\item There is a functor $F \colon \cat{Set} \to \cat{Vect}_F$ that maps each set $X$ to the freely generated vector space with basis $X$. Such a functor is said to be a \newterm{free functor}.

\item The functor $M_n \colon \cat{CRing} \to \cat{Mon}$ maps each commutative ring $R$ to the monoid $M_n(R)$ of $n \times n$ matrices over $R$, and each function $f \colon R \to S$ to the induced function $M_n(f) \colon M_n(R) \to M_n(S)$ defined by $M_n(f)([a_{ij}]) = [f(a_{ij})]$ for all matrices $[a_{ij}] \in M_n(R)$.
\end{enumerate}
\end{Example}

\begin{Exercise}
\begin{enumerate}
\item Let $F \colon \cat C \to \cat D$ be a functor and $f$ a morphism in $\cat C$. Prove that if $f$ is a section (respectively, a retraction) in $\cat C$, then $F(f)$ is a section (respectively, a retraction) in $\cat D$. Hence show that functors preserve isomorphisms.
\item Show, by means of a counterexample, that a functor need not \newterm{reflect} isomorphisms -- i.e. if $F \colon \cat C \to \cat D$ is a functor and $f$ is a morphism in $\cat C$ such that $F(f)$ is an isomorphism in $\cat D$, then $f$ need not be an isomorphism in $\cat C$.
\end{enumerate}
\end{Exercise}

\subsection{Hom Functors}

There are two types of \newterm{hom functors} from any locally small category into $\cat{Set}$. For any object $X$ of a locally small category $\cat C$, the \newterm{covariant hom functor} from $\cat C$ to $\cat{Set}$ defined by $X$ is the functor $\Hom(X, -) \colon \cat C \to \cat{Set}$ that maps each object $A$ of $\cat C$ to the set $\Hom(X, A)$, and each morphism $f \colon A \to B$ in $\cat C$ to the function $(f \circ -) \colon \Hom(X, A) \to \Hom(X, B)$ given by $(f \circ -)(h) = f \circ h$, for all $h \in \Hom(X, A)$.
\begin{equation*}
\xymatrix{
X \ar[r]^h \ar@{-->}[rd]_{f \circ h} & A \ar[d]^f \\
 & B
}
\end{equation*}

The \newterm{contravariant hom functor} from $\cat C$ to $\cat{Set}$ defined by $X$ is the functor $\Hom(-, X) \colon \cat C\opp \to \cat{Set}$ that maps each object $A$ of $\cat C$ to the set $\Hom(A, X)$, and each morphism $f \colon A \to B$ in $\cat C$ to the function $(- \circ f) \colon \Hom(B, X) \to \Hom(A, X)$ given by $(- \circ f)(h) = h \circ f$ for all $h \in \Hom(B, X)$.
\begin{equation*}
\xymatrix{
A \ar[d]_f \ar@{-->}[rd]^{h \circ f} & \\
B \ar[r]_h & X
}
\end{equation*}

\begin{Exercise}
Let $U \colon \cat{Ring} \to \cat{Set}$ be the forgetful functor that maps each unital ring $R$ to its underlying set $UR$. Find a family of functions $\{\phi_R\}$ satisfying the following:
\begin{enumerate}[label = (\roman*)]
\item For each unital ring $R$, $\phi_R \colon UR \to \Hom(\mathbb Z, R)$ is an isomorphism in $\cat{Set}$ (i.e. a bijection).
\item For each unital ring homomorphism $f \colon R\ to S$, the diagram below commutes.
\begin{equation*}
\xymatrix@=3em{
UR \ar[r]^f \ar[d]_{\phi_R} & US \ar[d]^{\phi_S} \\
\Hom(\mathbb Z, R) \ar[r]_{f \circ -} & \Hom(\mathbb Z, S)
}
\end{equation*}
\end{enumerate}
\end{Exercise}

\subsection*{Groups of Units and Group Rings}

Let $R$ be a unital ring. Then an element $u \in R$ is a \newterm{unit} if there exists an element $v \in R$ such that $uv = vu = 1$. The set of all units of $R$ is denoted by $R^\times$, and it is easy to see that it forms a group -- this group is the \newterm{group of units} of $R$.

\begin{Exercise}
\begin{enumerate}
\item Let $f \colon R \to S$ be a unital ring homomorphism from a unital ring $R$ to a unital ring $S$. Show that if $u \in R^\times$, then $f(u) \in S^\times$.
\item Show that the mapping of rings to their groups of units is functorial -- i.e. there exists a functor $(-)^\times \colon \cat{Ring} \to \cat{Grp}$ that maps each ring $R$ to $R^\times$, and each unital ring homomorphism $f \colon R \to S$ to a group homomorphism.

\hint{Define the image of $f$ under the functor as $f|_{R^\times}$. Verify that it is a group homomorphism from $R^\times$ to $S^\times$.}
\end{enumerate}
\end{Exercise}

Let $G$ be a group. Then the \newterm{group ring} of $G$ (over $\mathbb Z$) is the unital ring $\mathbb Z[G]$ defined as
\begin{equation*}
\mathbb Z[G] = \qty{\, a_1 g_1 + \cdots + a_n g_n \mid a_i \in R,\, g_i \in G\, i = 1, \ldots, n, \, n \in \mathbb N_0 \,}
\end{equation*}
the set of all formal finite linear combinations of elements of $G$ with coefficients from $R$, with addition and multiplication defined as follows: Let $x, y \in \mathbb Z[G]$. Then there exist $a_1, \ldots, a_n$ and $b_1, \ldots, b_n$ such that $x = \sum_{i=1}^n a_i g_i$ and $y = \sum_{i=1}^n b_i g_i$, where we take $a_i = 0$ if $g_i$ is not present in $x$ but is present in $y$, and similarly $b_i = 0$ if $g_i$ is present in $x$ but not present in $y$. Define
\begin{equation*}
\sum_{i=1}^n a_i g_i + \sum_{j=1}^n b_i g_i = \sum_{i=1}^n (a_i + b_i) g_i.
\end{equation*}
Now, define the product of $\sum_{i=1}^m a_i g_i$ and $\sum_{j = 1}^n b_j h_j$ as
\begin{equation*}
\pqty{\sum_{i=1}^m a_i g_i}\pqty{\sum_{j = 1}^n b_j h_j} = \sum_{i=1}^m \sum_{j=1}^n a_i b_j g_i h_j.
\end{equation*}

\begin{Exercise}
\begin{enumerate}
\item Verify that for any group $G$, the group ring $\mathbb Z[G]$ is a unital ring with unity $1e$, where $e$ is the identity element of $G$ and $1$ is the unity of $R$.
\item Prove that in $\mathbb Z[G]$, each element $g$ of $G$, considered as an element $1g \in \mathbb Z[G]$, is a unit.
\item Let $f \colon G \to H$ be a group homomorphism. Then show that the mapping $\sum_{i=1}^n a_i g_i \mapsto \sum_{i=1}^n a_i f(g_i)$ defines a unital ring homomorphism from $\mathbb Z[G]$ to $\mathbb Z[H]$.
\item Show that the mapping $\mathbb Z[-] \colon \cat{Grp} \to \cat{Ring}$ is functorial.
\item Show that for all groups $G$ and all rings $R$, there is an isomorphism of hom sets $\Hom(\mathbb Z[G], R) \cong \Hom(G, R^\times)$.
\end{enumerate}
\end{Exercise}

\bibliographystyle{plain}
\bibliography{References.bib}