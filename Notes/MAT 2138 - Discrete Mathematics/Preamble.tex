% !TeX root = MAT 2138 - Discrete Mathematics.tex

\usepackage{natbib}

\usepackage{amsmath, amsfonts, amssymb, amsthm, bbm}
\usepackage[all, cmtip, 2cell]{xy}
\setcounter{tocdepth}{3}
\usepackage{graphicx, subcaption}
\usepackage{physics}

\usepackage{tikz}

\usepackage{mathtools}
\DeclarePairedDelimiter{\ceil}{\lceil}{\rceil}
\DeclarePairedDelimiter{\floor}{\lfloor}{\rfloor}

\usepackage{xspace, cancel}

\usepackage{enumitem, array}
\setlist{noitemsep}

\usepackage{scrlayer-scrpage}
\ohead{\color{blue!35!black} \scshape VM}
\cfoot*{\pagemark}

\usepackage{tocloft}
\renewcommand{\cftdot}{}

\usepackage{hyperref}
\definecolor{linkcolor}{RGB}{32, 96, 192}
\hypersetup{colorlinks, linkcolor = linkcolor, urlcolor = linkcolor, linktocpage = true}
\usepackage{bookmark}
\bookmarksetup{color = [RGB]{32, 96, 192}}

\usepackage[capitalise]{cleveref}
\crefformat{enumi}{#2\textup{#1}#3}

\usepackage{kbordermatrix}
\usepackage{algorithm, algpseudocode}

\usepackage{ocgx2}

\usepackage[intoc]{nomencl}
\makenomenclature

\usepackage[toc, page]{appendix}

\usepackage{newpxmath}
\usepackage{charter}
\usepackage[T1]{fontenc}

\newtheorem{Theorem}{Theorem}[section]
\newtheorem{Lemma}[Theorem]{Lemma}
\newtheorem{Corollary}[Theorem]{Corollary}
\newtheorem{Observation}[Theorem]{Observation}

\theoremstyle{definition}
\newtheorem{Definition}[Theorem]{Definition}
\newtheorem*{Definition*}{Definition}
\newtheorem{Example}[Theorem]{Example}
\newtheorem*{Example*}{Example}
\newtheorem{Exercise}{Exercise}[section]

\theoremstyle{remark}
\newtheorem*{Remark*}{Remark}
\newtheorem*{Solution*}{Solution}
\newtheorem*{Note*}{Note}


\definecolor{alertcolor}{rgb}{0.8, 0.5, 0}
\newcommand{\newterm}[1]{{\color{alertcolor} #1}}

\definecolor{hintcolor}{rgb}{0, 0, 0.5}
\newcommand{\hint}[1]{{\tiny\color{hintcolor}\textbf{Hint:} #1}}

\DeclareMathOperator{\Nb}{N}
\DeclareMathOperator{\lcm}{lcm}
\DeclareMathOperator{\im}{im}
\DeclareMathOperator{\dom}{dom}
\DeclareMathOperator{\cod}{cod}
\DeclareMathOperator{\ecc}{ecc}
\DeclareMathOperator{\rad}{rad}
\DeclareMathOperator{\diam}{diam}
\newcommand{\id}{\mathrm{id}}
\newcommand{\symdiff}{\mathbin{\triangle}}
\newcommand{\nth}{\textsuperscript{th}\xspace}
\DeclareMathOperator{\bestDTo}{bestDTo}
\DeclareMathOperator{\tree}{tree}

\algrenewcommand\algorithmicrequire{\textbf{Input:}}