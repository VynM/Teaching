% !TeX root = MAT 2138 - Discrete Mathematics.tex

\title{\textbf{Discrete Mathematics}}

\date{}
\maketitle

\begingroup
\let\clearpage\relax
\tableofcontents
\endgroup

\clearpage

\renewcommand{\nomname}{List of Symbols}
\nomenclature[01]{$\forall$}{For all/any}
\nomenclature[02]{$\exists$}{For some / There exists}
\nomenclature[03]{$a \in S$}{$a$ belongs to / is an element of set $S$}
\nomenclature[04]{$\cup$, $\cap$}{Union and intersection (respectively) of sets}
\nomenclature[05]{$\varnothing$}{Empty set $\{\}$}
\nomenclature[06]{$\mathbb N$, $\mathbb Z$, $\mathbb Q$, $\mathbb R$, $\mathbb C$}{Sets of natural numbers, integers, rational numbers, real numbers, complex numbers (respectively)}
\nomenclature[07]{$\mathbb N_0$}{Set of non-negative integers $\{0, 1, 2, \ldots\}$}
%\nomenclature[08]{$\mathbb R_{>0}$}{Set of positive real numbers}
\nomenclature[08]{$A \subseteq B$}{$A$ is a subset of set $B$}
\nomenclature[09]{$\lvert S \rvert$}{Cardinality of set $S$}
\printnomenclature[10em]

\clearpage

\section{Elementary Set Theory}\label{sec:SetTheory}

A \newterm{set} is (informally) an unordered collection of \newterm{elements}. More formally, any set is defined by the \newterm{membership relation} $\in$ (read ``belongs to'', ``is an element of'', ``is a member of'', or ``is in''), where $s \in S$ if and only if $s$ is an element of the set $S$. If $x$ is not a member of $S$, then we write $x \notin S$. We may define a set either as a list of all its members enclosed by curly braces -- $\{$ and $\}$ -- or in the form $Y = \qty{\, x \in X \mid P(x) \,}$, where $X$ is a previously defined set, and $P(x)$ is a \newterm{predicate} (a function with a true/false value depending on the value of $x$) -- then $Y$ is the set consisting of all members of $X$ that satisfy the predicate $P(x)$ (i.e. those $x \in X$ for which $P(x)$ is true). The latter form of defining a set is called the \newterm{set-builder} notation. For example
\begin{equation*}
S = \qty{1, \text{`a'}, 2.5, +, 9, -3}
\end{equation*}
defines $S$ to be the set consisting of the six elements $1$, `a', $2.5$, $+$, $9$, and $-3$, and
\begin{equation*}
T = \qty{\, s \in S \mid s\ \text{is a number} \,}
\end{equation*}
defines $T$ to be the set consisting of the four elements $1$, $2.5$, $9$, and $-3$. The symbol {$|$} is read as ``such that'' (and can be replaced by a $:$ as well).

The \newterm{universal set} (usually denoted by $U$) is the set consisting of all elements currently under consideration. We may write $\qty{\, x \mid P(x) \,}$ to mean $\qty{\, x \in U \mid P(x) \,}$.

The \newterm{empty set} (or \newterm{null set}) is the set $\varnothing$ that has no elements. That is, for every element $a$ of the universal set, $a \notin \varnothing$.

\begin{Note*}
A variant of the set-builder notation replaces the element on the left side of $\mid$ by an expression involving one or more elements, with the specifications of memberships of these elements appearing on the right side of $\mid$, along with other predicates, if any. For example
\begin{equation*}
S = \qty{\, 2n \mid n \in \mathbb Z \,}
\end{equation*}
defines $S$ to be the set of all elements obtained by doubling an integer -- in other words, $S$ is the set of even integers.
\end{Note*}

\subsection{Relations Among Sets}\label{subsec:SetRels}

A set $A$ is a \newterm{subset} of a set $B$, denoted by $A \subseteq B$, if every element of $A$ is an element of $B$. That is, for any element $a$ (of the universal set), $a \in A \implies a \in B$. Then $B$ is a \newterm{superset} of $A$, denoted by $B \supseteq A$. We may also say that $B$ \newterm{contains} $A$, or that $A$ is contained in $B$.

Two sets $A$ and $B$ are \newterm{equal}, written $A = B$, if each contains the other -- i.e. $A \subseteq B$ and $B \subseteq A$. This is equivalent to the statement that $A$ and $B$ have exactly the same elements. Otherwise, $A$ is not equal to $B$ (written $A \ne B$). $A$ is a \newterm{proper} subset of $A$ (or is properly contained in $B$) if $A \subseteq B$ and $A \ne B$. Then we write $A \subsetneq B$. Similarly, $B$ is a proper superset of $A$, denoted by $B \supsetneq A$, if $B \supseteq A$ and $B \ne A$.

\begin{Note*}
It is also common to use $\subset$ and $\supset$ instead of $\subseteq$ and $\supseteq$, respectively. They are usually \emph{not} alternatives to $\subsetneq$ and $\supsetneq$, except when explicitly stated to be so.
\end{Note*}

The set of all subsets of a set $A$ is called the \newterm{power set} of $A$, is denoted by $2^A$ or $\mathcal P(A)$. That is,
\begin{equation*}
2^A = \qty{\, S \mid S \subseteq A \,}.
\end{equation*}

\begin{Exercise}
Let $A$, $B$, and $C$ be arbitrary sets.
\begin{enumerate}
\item Show that $A \subseteq A$.
\item Show that if $A \subseteq B$ and $B \subseteq C$, then $A \subseteq C$.
\item Show that the $\supseteq$ also satisfies these properties.
\item Show that $\varnothing \subseteq A$.
\item Show that if $A$ is a set consisting of $n$ elements, for some non-negative integer $n$, then $2^A$ contains $2^n$ elements.
\end{enumerate}
\end{Exercise}

\subsection{Basic Operations of Sets}\label{subsec:BasicSetOps}

The \newterm{union} of two sets $A$ and $B$ is the set $A \cup B$ consisting of all elements that belong to $A$ or $B$:
\begin{equation*}
A \cup B = \qty{\, x \mid x \in A\ \text{or}\ x \in B\,}.
\end{equation*}
The \newterm{intersection} of two sets $A$ and $B$ is the set $A \cap B$ consisting of all elements that belong to $A$ and $B$:
\begin{equation*}
A \cap B = \qty{\, x \mid x \in A\ \text{and}\ x \in B\,}.
\end{equation*}
The \newterm{complement} of a set $A$ is the set $\overline A$ of all elements (of the universal set) that do not belong to $A$:
\begin{equation*}
\overline A = \qty{\, x \mid x \notin A \,}.
\end{equation*}
We may also denote the complement of $A$ by $A'$, $A^{\mathrm c}$, or $U \setminus A$.

\begin{Exercise}
Let $A$, $B$, and $C$ be arbitrary sets.
\begin{enumerate}
\item Show that $A \cap B \subseteq A \subseteq A \cup B$.

\item Show that $\cup$ is
\begin{enumerate}[label=(\roman*)]
\item \newterm{associative}: $(A \cup B) \cup C = A \cup (B \cup C)$,
\item \newterm{commutative}: $A \cup B = B \cup A$, and
\item \newterm{idempotent}: $A \cup A = A$.
\end{enumerate}

\item Prove that $\cap$ is also associative, commutative, and idempotent.

\item Prove that $\cup$ \newterm{distributes} over $\cap$ and vice-versa:
\begin{align*}
A \cup (B \cap C) & = (A \cup B) \cap (A \cup C) \\
A \cap (B \cup C) & = (A \cap B) \cup (A \cap C).
\end{align*}

\item Prove that $\cup$ and $\cap$ satisfy the law of \newterm{absorption}:
\begin{equation*}
A \cup (A \cap B) = A \cap (A \cup B) = A.
\end{equation*}

\item Show that the following are equivalent:
\begin{enumerate}
\item $A \subseteq B$.
\item $A \cup B = B$.
\item $A \cap B = A$.
\end{enumerate}

\item Prove that $\cup$, $\cap$ and $\overline{~}$ satisfy De Morgan's laws. That is:
\begin{align*}
\overline {\pqty{A \cup B}} & = \overline A \cap \overline B \\
\overline {\pqty{A \cap B}} & = \overline A \cup \overline B.
\end{align*}

\item Show that $A \cap \overline A = \varnothing$.
\end{enumerate}

\end{Exercise}

Two sets $A$ and $B$ are \newterm{disjoint} if their intersection is empty -- i.e. $A \cap B = \varnothing$. Note that $A$ and $\overline A$ are always disjoint. Also note that $\varnothing$ is disjoint with all sets, and is the unique set that is disjoint with itself.

Since $\cup$ and $\cap$ are associative, expressions of the form $A_1 \cup A_2 \cup \cdots \cup A_n$ and $A_1 \cap A_2 \cap \cdots \cap A_n$ are well-defined and unambiguous. These are called the \newterm{$n$-ary} (or finite) union and intersection, and denote them as $\bigcup_{i=1}^n A_i$ and $\bigcap_{i=1}^n A_i$, respectively. But it is possible to define unions and intersections of collections of sets even more generally. For this, we will discuss the concept of an indexing set.

\subsection{Index Sets}\label{subsec:IndexSets}

A set $I$ is an \newterm{index set} (or \newterm{indexing set}) of a set $S$ if we can write $S$ as
\begin{equation*}
S = \{ s_i \mid i \in I\}.
\end{equation*}
That is, each element of $s_i \in S$ corresponds to a unique element $i \in I$. Then $S$ is \newterm{indexed by} $I$, and it is common to write $S = \{ s_i \}_{i \in I}$.

Let $\mathcal A = \{ A_i \}_{i \in I}$ be a collection of sets (for some index set $I$). Thus, for each $i \in I$, $A_i$ itself is a set in the collection $\mathcal A$. Then we can define the union and intersection of the sets in $\mathcal A$, as given below.
\begin{align*}
\bigcup_{i \in I} A_i & = \qty{\, a \mid a \in A_i,\, \text{for some}~i \in I \,} \\
\bigcap_{i \in I} A_i & = \qty{\, a \mid a \in A_i,\, \text{for all}~ i \in I \,}
\end{align*}
We refer to these operations as \newterm{arbitrary} unions and intersections. In the particular case where $I$ is a set containing finitely many elements, these reduce to the finite union and intersection defined earlier.

\subsection{Some More Set Operations}\label{subsec:MoreSetOps}

We say that $(a, b)$ is an \newterm{ordered pair}\footnote{Formally, we may define the ordered pair in terms of sets as $(a, b) = \qty{a, \qty{a, b}}$. Note that this is only one possible ``encoding'' of the concept of an ordered pair, and in practice, we do not think of $(a, b)$ as the (unordered) set $\qty{a, \qty{a, b}}$.} where the first element is $a$ and the second element is $b$.

The \newterm{Cartesian product} of two sets $A$ and $B$, denoted by $A \times B$, is the set of all ordered pairs of elements with the first element from $A$ and the second from $B$. That is,
\begin{equation*}
A \times B = \qty{\, (a, b) \mid a \in A,\, b \in B \,}.
\end{equation*}
The Cartesian product of $A$ with itself is often written as $A^2$. Similarly, the Cartesian product $A \times \cdots A$ with $n$ terms (defined as an iterated Cartesian product of two sets at a time) is denoted by $A^n$. 

The \newterm{disjoint union} (or \newterm{coproduct}) of two sets $A$ and $B$, denoted by $A \sqcup B$, consists of all the ordered pairs of the form $(x, i)$ where $i = 1$ when $x \in A$ and $i = 2$ when $x \in B$. That is,
\begin{equation*}
A \sqcup B = \qty{\, (a, 1) \mid a \in A \,} \cup \qty{\, (b, 2) \mid b \in B \,}.
\end{equation*}
The disjoint union is also denoted by $A \mathbin{\mathaccent\cdot\cup} B$, or $A \uplus B$.

\begin{Note*}
The second element in each ordered pair (i.e. $1$ or $2$) only serves to distinguish the elements that are originally from $A$, from those that are originally from $B$. Thus, for example, if $x$ is an element common to both $A$ and $B$, then $A \sqcup B$ contains two ``copies'' of $x$, namely $(x, 1)$ and $(x, 2)$. When $A$ and $B$ are disjoint, $A \sqcup B$ is equivalent to $A \cup B$ (where the meaning of ``equivalent'' will be formalised later).
\end{Note*}

We can also define Cartesian products and disjoint unions of a collection of sets. Let $\mathcal A = \{A_i\}_{i \in I}$ be a collection of sets indexed by $I$. Then the Cartesian product of the collection $\mathcal A$ is
\begin{equation*}
\prod_{i \in I} A_i = \qty{\, (a_i)_{i \in I} \mid a_i \in A_i,~ i \in I \,}
\end{equation*}
where $(a_i)_{i \in I}$ is a sequence of elements indexed by $I$, with $a_i \in A_i$ for each $i \in I$. The disjoint union of the collection $\mathcal A$ is
\begin{equation*}
\bigsqcup_{i \in I} A_i = \bigcup_{i \in I} \qty{\, (a_i, i) \mid a_i \in A_i,~ i \in I \,} = \bigcup_{i \in I} A_i \times \{i\}.
\end{equation*}

\begin{Note*}
We will formally define sequences later, in \cref{subsec:Functions}.
\end{Note*}

The \newterm{difference} of two sets $A$ and $B$ is the set $A \setminus B$ consisting all elements of $A$ that are not elements of $B$. That is,
\begin{equation*}
A \setminus B = \{ a \in A \mid a \notin B \}.
\end{equation*}
The difference of $A$ and $B$ is also denoted by $A - B$. Note that $A \setminus B = A \cap \overline B = A \setminus (A \cap B)$.

The \newterm{symmetric difference} of two sets $A$ and $B$ is the set $A \symdiff B$ consisting of all the elements that are present in exactly one of $A$ and $B$. That is,
\begin{equation*}
A \symdiff B = (A \setminus B) \cup (B \setminus A).
\end{equation*}
The symmetric difference of $A$ and $B$ is also denoted by $A \ominus B$.

\begin{Exercise}
Let $A$ and $B$ be arbitrary sets.
\begin{enumerate}
\item Show that $(A \cup B) \setminus B = A \setminus B$.
\item Show that $A \symdiff B = (A \cup B) \setminus (A \cap B)$.
\item Show that $(2^A, \symdiff)$ is an Abelian group. That is:
    \begin{enumerate}[label=(\roman*)]
    \item $\symdiff$ is associative.
    \item $\symdiff$ is commutative.
    \item There exists $E \in 2^A$ such that for all $S \in 2^A$, $S \symdiff E = S$.
    \item For all $S \in 2^A$, there exists $T \in 2^A$, such that $S \symdiff T = E$.
    \end{enumerate}
    What is the order of any non-identity element of this group?
\end{enumerate}
\end{Exercise}

\subsection{Relations}\label{subsec:Relations}
A \newterm{relation} $R$ from a set $A$ to a set $B$, denoted $R \colon A \to B$, is a subset of $A \times B$, i.e. $R \subseteq A \times B$. If $(a, b) \in R$, then we write $aRb$, and if $(a, b) \notin R$, then we write $a \cancel{R} b$. The set $A$ is the \newterm{domain} and $B$ the \newterm{codomain} of $R$. Note that $\varnothing$ is also a relation, called the \newterm{empty} or \newterm{void} relation, from $A$ to $B$.

\begin{Note*}
A relation from a set $A$ to a set $B$ is a \newterm{binary relation}. More generally, if $A_1, \ldots, A_n$ are $n$ sets, then a subset of $A_1 \times \cdots \times A_n$ is an \newterm{$n$-ary relation}.
\end{Note*}

A relation $R \colon A \to B$ is said to be
\begin{enumerate}
\item \newterm{left-total} if for each $a \in A$, $aRb$ for some $b \in B$.
\item \newterm{right-total} if for each $b \in B$, $aRb$ for some $a \in A$.
\item \newterm{left-unique} if for each $b \in B$, if $a, a' \in A$ are such that $aRb$ and $a'Rb$, then $a = a'$.
\item \newterm{right-unique} if for each $a \in A$, if $b, b' \in B$ are such that $aRb$ and $aRb'$, then $b = b'$.
\end{enumerate}

A relation from $A$ to itself is said to be a \newterm{relation on} (or \newterm{over}) $A$ (also called a \newterm{homogeneous} relation on $A$). Such relations can have a number of properties. In the following, let $\sim$ be a relation on a set $A$.
\begin{enumerate}
\item \newterm{Reflexivity}: For all $a \in A$, $a \sim a$.
\item \newterm{Symmetry}: For all $a, b \in A$, if $a \sim b$, then $b \sim a$.
\item \newterm{Anti-symmetry}: For all $a, b \in A$, if $a \sim b$ and $b \sim a$, then $a = b$.
\item \newterm{Transitivity}: For all $a, b, c \in A$, if $a \sim b$ and $b \sim c$, then $a \sim c$.
\item \newterm{Irreflexivity}: For all $a \in A$, $a \not\sim a$.
\item \newterm{Asymmetry}: For all $a \in A$, if $a \sim b$, then $b \not\sim a$.
\end{enumerate}

\begin{Example}
The following list gives examples of familiar relations satisfying one or more of the above properties:
\begin{enumerate}
\item The relations $=$ (on any set of elements where equality is defined), $\le$ and $\ge$ (on any set of real numbers), $\mid$ (\emph{divides}, see \cref{exer:DividesRel}), $\subseteq$ and $\supseteq$ (on any collection of sets), $\cong$ and $\sim$ (on any set of triangles), and $\parallel$ (on any set of lines) are reflexive.
\item The relations $=$, $\ne$, $\cong$, $\sim$, $\parallel$, $\perp$ are symmetric.
\item The relations $\mid$ (on any set of non-negative integers), $\le$, $\ge$, $\subseteq$, $\supseteq$ are anti-symmetric.
\item The relations $=$, $\le$, $\ge$, $<$, $>$, $\mid$, $\subseteq$, $\supseteq$, $\subsetneq$, $\supseteq$, $\cong$, $\sim$, $\parallel$ are transitive.
\item The relations $\ne$ and $\perp$ are irreflexive.
\item The relations $<$, $>$, $\subsetneq$, $\supsetneq$ are asymmetric.
\end{enumerate}
\end{Example}

\begin{Exercise}~
\begin{enumerate}
\item Let $\sim$ be the relation of the set $A = \{a, b\}$ defined by $a \sim a$, $a \sim b$. Is $\sim$ transitive?
\item Show that every asymmetric relation is irreflexive.
\item Prove or disprove: Any transitive, irreflexive relation is asymmetric.
\item Prove or disprove: Any symmetric, transitive relation is reflexive.
\end{enumerate}
\end{Exercise}

\subsubsection{Equivalence Relations and Partitions}

A reflexive, symmetric, and transitive relation is called an \newterm{equivalence} relation. For example, $=$, $\cong$, $\sim$, and $\parallel$ are equivalence relations. If $\sim$ is an equivalence relation on a set $A$, and $a \in A$, then the \newterm{equivalence class} of $a$, denoted as $[a]$ or $\overline a$ is the set of all elements of $A$ that $a$ is related to by $\sim$. That is,
\begin{equation*}
[a] = \qty{\, b \in A \mid a \sim b \,}.
\end{equation*}

\begin{Example}
Let $A = \{1, 2, 3, 4, 5\}$, and define a relation $\sim$ on $A$ as follows: For any $a, b \in A$, $a \sim b$ if and only if $a - b$ is even. Then, for example, the equivalence class of $1$ is $[1] = \{1, 3, 5\}$, and the equivalence class of $2$ is $[2] = \{2, 4, 6\}$. Note that $[1] = [3] = [5]$ and $[2] = [4] = [6]$. Also observe that $[1]$ and $[2]$ are disjoint, and $[1] \cup [2] = A$.
\end{Example}

An equivalence relation on a set is essentially the same as a partition of the set, as you will show in \cref{exer:EqRel->Partition,exer:Partition->EqRel}. A \newterm{partition of a set} $S$ is a collection of non-empty and pairwise disjoint subsets of $S$ whose union is equal to $S$. That is, a partition of $S$ is a collection $\{P_i\}_{i \in I}$ of sets $P_i \subseteq S$, $i \in I$, such that
\begin{enumerate}
\item $P_i \ne \varnothing$, for each $i \in I$,
\item $P_i \cap P_j = \varnothing$, for all $i, j \in I$, $i \ne j$, and
\item $\bigcup_{i \in I} P_i = S$.
\end{enumerate}
The subsets $P_i$, $i \in I$, are called the \newterm{parts} of the partition $P$.

\begin{Example}
Let $S = \{1, 2, 3, 4, 5\}$. Then $P = \{\{1, 3, 5\}, \{2, 4, 6\}\}$ and $Q = \{\{1, 4\}, \{2\}, \{3, 5\}\}\}$ are two different partitions of $S$.
\end{Example}

\begin{Exercise}\label{exer:EqRel->Partition}
Let $\sim$ be an equivalence relation on a set $A$. Show that the following hold:
\begin{enumerate}
\item For all $a \in A$, $a \in [a]$, and hence, each equivalence class is non-empty and $A = \bigcup_{a \in A} [a]$.
\item For all $a, b \in A$, if $a \ne b$, then $[a] \cap [b] = \varnothing$ (i.e. any two equivalence classes are either disjoint or identical).
\item The set of all equivalence classes of $\sim$ is a partition of $A$.
\end{enumerate}
\end{Exercise}

\begin{Exercise}\label{exer:Partition->EqRel}
Let $P = \{P_i\}_{i \in I}$ be a partition of a set $A$. Define a relation $\sim$ on $A$ as follows: For any $a, b \in A$, $a \sim b$ if and only if $a$ and $b$ belong to the same part of the partition $P$ (i.e. $a, b \in P_i$, $\exists i \in I$). Show that the following hold:
\begin{enumerate}
\item The relation $\sim$ is an equivalence relation on $A$.
\item The equivalence classes of $\sim$ are exactly the parts of the partition of $P$.
\end{enumerate}
\end{Exercise}

\subsubsection{Partial Order Relations}

A reflexive, anti-symmetric, and transitive relation is called a \newterm{partial order} relation (or simply a partial order). For example, $\le$ and $\ge$ on any set of real numbers, $\mid$ on any set of non-negative integers, and $\subseteq$ and $\supseteq$ on any set of sets are partial order relations. A set $A$ together with a partial order $\preccurlyeq$ on it forms a \newterm{partially ordered set} or \newterm{poset} $(A, \preccurlyeq)$.

\begin{Note*}
The term \emph{partial} refers to the fact that two particular elements in a partially ordered set may be \newterm{incomparable} -- i.e. neither may be related to the other in the partial order. For instance, consider the subsets of $S = \{x, y, z\}$, which are partially ordered by the subset relation $\subseteq$ -- i.e. consider the poset $(2^S, \subseteq)$. Then $A = \{x, y\}$ and $B =\{y, z\}$ are incomparable, as neither is $A$ a subset of $B$, nor is $B$ a subset of $A$. On the other hand, $A$ and $C = \{x\}$ are \newterm{comparable} (as $C \subseteq A$), and $A$ and $S$ itself are also comparable (as $A \subseteq S$). A poset in which every pair of elements is comparable (i.e. in which there are no incomparable pairs of elements) is called a \newterm{total order}.
\end{Note*}

\begin{Exercise}\label{exer:DividesRel}
Let $\mid$ denote the \newterm{divides} relation on any set of integers. That is, for any two integers $m$ and $n$, define $m \mid n$ if and only if $n = km$ for some integer $k$.
\begin{enumerate}
\item Prove that $(\mathbb N, \mid)$ is a poset. Is $(\mathbb N_0, \mid)$ also a poset?
\item Is $(\mathbb Z, \mid)$ a poset?
\item Let $n \in \mathbb N$, and let $P$ be the set of all positive divisors of $n$. Then show that $(P, \mid)$ is a poset. What are all the natural numbers $n$ such that $(P, \mid)$ is a total order?
\end{enumerate}
\end{Exercise}

\subsection{Functions}\label{subsec:Functions}

A \newterm{function} is a left-total, right-unique binary relation. In other words, a function $f \colon A \to B$ is a relation from $A$ to $B$ such that each element of $A$ is related to exactly one element of $B$ under $f$. If $a \in A$ is related to $b \in B$ in $f$, then we say that $f$ \newterm{maps} $a$ to $b$, and write $b = f(a)$, or $a \mapsto b$. A function is also called a \newterm{mapping}. Recall that $A$ is the domain of $f$ and $B$ the codomain. We may also write $\dom f$ and $\cod f$ to denote the domain and codomain of $f$, respectively. The \newterm{image} of $f$, denoted as $\im f$ or $f(A)$, is the set of all elements $b$ of the codomain such that $b = f(a)$ for some $a \in A$. We can write this in the following two ways:
\begin{align*}
\im f & = \qty{\, b \in B \mid b = f(a),\, \text{for some}~ a \in A \,} \\
\im f & = \qty{\, f(a) \mid a \in A \,}.
\end{align*}
The \newterm{preimage} of any element of the codomain is the set of all elements of the domain that map to it. That is, for $b \in B$, the preimage of $b$, denoted $f^{-1}(b)$, is defined as
\begin{equation*}
f^{-1}(b) = \qty{\, a \in A \mid f(a) = b \,}.
\end{equation*}
Note that the image of the function is a subset of the codomain, while the preimage of an element is the subset of the domain.

\begin{Note*}
Intuitively, we think of a function as a rule that assigns, to each element of the domain, a unique element of the codomain. For instance, it is common in calculus to define a function using a formula -- e.g. $f \colon \mathbb R \to \mathbb R$, $f(x) = x^2 - 1$. However, the formula or the expression itself is not the function. The function $g \colon \mathbb Z \to \mathbb Q$, defined by the formula $g(x) = x^2 - 1$, is different from the previously defined function $f$, although they are both defined using the same formula. Moreover, it may not be possible to define a function using any closed-form formula.
\end{Note*}

A function is \newterm{injective} (or \newterm{$1$-$1$}) if it is left-unique. That is, $f \colon A \to B$ is injective if, for any $a_1, a_2 \in A$, $f(a_1) = f(a_2)$ implies that $a_1 = a_2$. An injective function is also called an \newterm{injection}. A \newterm{surjective} (or \newterm{onto}) function is one in which every element of the codomain has a non-empty preimage. That is, $f \colon A \to B$ is surjective if, for each $b \in B$, $b = f(a)$ for some $a \in A$. Note that $f$ is surjective if and only if $\im f = \cod f$. A surjective function is also called a \newterm{surjection}. A function that is both injective and surjective is \newterm{bijective}. A bijective function is also called a \newterm{bijection} or a \newterm{one-to-one correspondence}.

\begin{Example}
Let $A = \{1, 2, 3, 4, 5\}$ and $B = \{a, b, c\}$.
\begin{enumerate}
\item Define a function $f \colon A \to B$, $f(1) = f(2) = a$, $f(3) = b$, $f(4) = f(5) = c$. Then $f$ is a surjection ($a$ has preimage $\{1, 2\}$, $b$ has preimage $\{3\}$, and $c$ has preimage $\{4, 5\})$. It is clearly not an injection, since, for example, $f(1) = f(2)$.

\item Define $g \colon B \to A$, $g(a) = 1$, $g(b) = 5$, $g(c) = 4$. Then $g$ is an injection from $B$ to $A$, as $g(a) \ne g(b), g(c)$ and $g(b) \ne g(c)$. The image of $g$ is $\im g = \{1, 4, 5\} \ne A = \cod g$, and hence $g$ is not a surjection.

\item Define a $h \colon B \to B$, $h(a) = b$, $h(b) = c$, $h(c) = a$. Note that $h$ is a bijection from $B$ to itself.
\end{enumerate}
\end{Example}

\begin{Example}
Define $f \colon \mathbb Z \to \mathbb R$ as $f(n) = n$, for all $n \in \mathbb Z$. Then $f$ is an injection, but not a surjection (e.g. $1.5 \in \mathbb R$ has no preimage under $f$). Define $g \colon \mathbb R \to \mathbb Z$ as $g(x) = \lceil x \rceil$, the ceiling of $x$ (i.e. the smallest integer greater than or equal to $x$ -- e.g. $\lceil 3.2 \rceil = 4$, and $\lceil -1.5 \rceil = 0$). Then $g$ is a surjection (for any $n \in \mathbb Z$, $n \in \mathbb R$ as well, and $g(n) = \lceil n \rceil = n$), but not an injection (e.g. $g(1.8) = g(2) = 2$). Similarly, define $h \colon \mathbb R \to \mathbb Z$ as $h(x) = \lfloor x \rfloor$, the floor of $x$ (i.e. the greatest integer less than or equal to $x$). Then $h$ is a surjection (but not an injection) from $\mathbb R$ to $\mathbb Z$, different from $g$.
\end{Example}

\begin{Example}\label{ex:RIsom01}
Define $f \colon \mathbb R \to \mathbb (0,1)$ (the set of all real numbers strictly between $0$ and $1$) as
\begin{equation*}
f(x) = \dfrac{1}{1 + e^x}.
\end{equation*}
Firstly, note that this is indeed a well-defined function from $\mathbb R$ to $(0, 1)$, since $e^x \ge 0$ for all $x \in \mathbb R$. Now, if $f(x) = f(y)$, then observe that $e^x = e^y$, or $e^{x - y} = 1$, which implies that $x = y$. Hence, $f$ is injective. Next, let $y$ be any element of the codomain, $(0, 1)$. Then observe that $\frac 1 y > 1$, and hence $\frac 1 y - 1$ is a positive real number. Take $x = \log\pqty{\frac 1 y - 1}$, so that $\frac 1 {1 + e^x} = y$, i.e. $f(x) = y$. Thus, for every $y \in (0, 1)$, there exists $x \in \mathbb R$ such that $f(x) = y$, which shows that $f$ is surjective. Therefore, $f$ is a bijection from $\mathbb R$ to $(0, 1)$.
\end{Example}

\begin{Example}
Let $A = \{a_1, a_2, \ldots, a_n\}$, and let $P = 2^A$, the power set of $A$. Let $X$ be the set of all binary strings of length $n$ -- i.e. the set of all sequences of the form $x_1 x_2 \cdots x_n$, where $x_i \in \{0, 1\}$, $i = 1, \ldots, n$. Define a function $\chi \colon P \to X$ as follows: For each $S \in P$, $\chi(S) = b_1 b_2 \cdots b_n$ such that $b_i = 1$ if $a_i \in S$ and $b_i = 0$ if $a_i \notin S$. Then $\chi$ is a bijection, as shown below.

First, suppose that for $S, T \in P$, $\chi(S) = \chi(T) = b_1 b_2 \cdots b_n$ (say). Then, for each $i = 1, \ldots, n$, $a_i \in S$ if and only if $b_i = 1$, which is equivalent to $a_i \in T$. Thus, $S = T$. This shows that $\chi$ is injective.

Next, let $b_1 b_2 \cdots b_n \in X$. Define $S \subseteq A$ as
\begin{equation*}
S = \qty{\, a_i \in A \mid b_i = 1 \,}.
\end{equation*}
Then clearly, $\chi(S) = b_1 b_2 \cdots b_n$. Hence, $\chi$ is surjective.
\end{Example}

\subsection{Function Composition}\label{subsec:FuncComp}

If $f \colon A \to B$ and $g \colon B \to C$ are two functions, then the composition of $f$ and $g$ is the function $h \colon A \to C$ defined by $h(a) = g(f(a))$, for all $a \in A$. We denote this function $h$ by $g \circ f$, which is read as ``$g$ circle $f$'', ``$g$ after $f$'', or ``$g$ composed with $f$''. We can also write this definition in terms of a \newterm{commutative diagram}. Consider the diagram given below.
\begin{equation*}
\xymatrix{
    A \ar[r]^f \ar[rd]_h & B \ar[d]^g \\
    & C
}
\end{equation*}
We say that such a diagram \newterm{commutes} if the result of following any two directed paths from the same starting point to the same end point is the same. In the above diagram, $A, f, B, g, C$ and $A, h, C$ are two directed paths from $A$ to $C$ (and these are the only two directed paths with the same starting points and the same end points). Thus, the diagram commutes if and only if applying $g$ after $f$ equals $h$.

\begin{Theorem}\label{thm:FuncCompAssoc}
Composition of functions is an associative operation.
\end{Theorem}

\begin{proof}
To prove that two functions are equal, we need to show that they both map each element of the domain to the same element of the codomain. Consider three functions $f \colon A \to B$, $g \colon B \to C$, and $h \colon C \to D$. Let $a \in A$. We will show that $h \circ (g \circ f)$ and $(h \circ g) \circ f$ map $a$ to the same element of $C$.
\begin{align*}
(h \circ (g \circ f))(a) & = h((g \circ f)(a)) \\
    & = h(g(f(a))) \\
    & = (h \circ g)(f(a)) \\
    & = ((h \circ g) \circ f)(a).
\end{align*}
Hence, $h \circ (g \circ f) = (h \circ g) \circ f$.
\end{proof}

\begin{Exercise}
Show that in the diagram below, the square commutes if both the triangles commute.
\begin{equation*}
\xymatrix{
A \ar[r]^f \ar[d] & B \ar[dl]|-g \ar[d]\\
C \ar[r]_h & D
}
\end{equation*}
\end{Exercise}

The \newterm{identity function} on a set $A$, denoted as $\id_A$, is the function from $A$ to itself that maps every element of $A$ to itself. That is, $\id_A \colon A \to A$ is defined as $\id_A(a) = a$, for all $a \in A$. Note that $\id_A$ is a bijection from $A$ to itself. The identity function is so named because it is the identity element (or neutral element) for the operation of function composition. That is, if $f \colon A \to B$ is any function, then $f \circ \id_A = f$, and $\id_B \circ f = f$. Alternatively, we can say that the diagram given below commutes.
\begin{equation*}
\xymatrix{
A \ar[d]_{f} \ar[r]^{\id_A} \ar[dr]|-f & A \ar[d]^{f}\\
B \ar[r]_{\id_B} & B
}
\end{equation*}

\begin{Exercise}
Let $A$ be a set.
\begin{enumerate}
\item Show that $\id_A \circ \id_A = \id_A$.
\item Show that if $i \colon A \to A$ is any function satisfying the same property as the identity function, i.e. for every set $B$ and every function $f \colon A \to B$, $f \circ i = f$, and for every set $C$ and every function $g \colon C \to A$, $i \circ g = g$,  then $i = \id_A$.
\end{enumerate}
\end{Exercise}

\subsection{Isomorphisms of Sets}\label{subsec:SetIsos}
If $f \colon A \to B$ is a function, then a function $g \colon B \to A$ is an \newterm{inverse} of $f$ if $g \circ f = \id_A$ and $f \circ g = \id_B$. Equivalently, $g$ is an inverse of $f$ if the diagram given below commutes.
\begin{equation*}
\xymatrix{
A \ar[d]_{\id_A} \ar[dr]|-f & B \ar[d]^{\id_B} \ar[l]_g\\
A & B \ar[l]^g
}
\end{equation*}

\begin{Exercise}\label{exer:UniqueInverse}
Prove that any function has at most one inverse. \\[0.5em]
\hint{Assume that a function $f \colon A \to B$ has two inverses $g$ and $h$, and evaluate $g \circ f \circ h$ in two different ways.}
\end{Exercise}

A function $f \colon A \to B$ is an \newterm{isomorphism} (of sets) if it has an inverse. Then the unique inverse (see \cref{exer:UniqueInverse}) of $f$ is denoted by $f^{-1}$. Note that $f^{-1}$ is an isomorphism from $B$ to $A$. For any two sets $A$ and $B$, we say that $A$ \newterm{is isomorphic to} $B$ if there is an isomorphism from $A$ to $B$. We write this as $A \cong B$.

\begin{Exercise}
\mbox{}
\begin{enumerate}
\item Show that the identity function on a set is an isomorphism from the set to itself.
\item Prove that if $f$ and $g$ are isomorphisms, from $A$ to $B$ and from $B$ to $C$, respectively, then $g \circ f$ is an isomorphism from $A$ to $C$.
\item Show that $\cong$ is an equivalence relation on any collection of sets.
\end{enumerate}
\end{Exercise}

Now, we will prove that an isomorphism of sets is exactly the same thing as a bijection.

\begin{Theorem}
A function $f \colon A \to B$ is an isomorphism if and only if it is a bijection.
\end{Theorem}

\begin{proof}
First, suppose that $f$ is a bijection. We need to show that $f$ is an isomorphism from $A$ to $B$ -- i.e. that it has an inverse. We construct the inverse as follows. Define $g \colon B \to A$ as $g(b) = a$, where $a \in A$ is such that $f(a) = b$, for each $b \in B$.

To see that $g$ is well-defined, observe that as $f$ is bijective, for every $b \in B$, there exists an $a \in A$ such that $f(a) = b$, and as $f$ is injective, this element $a$ is unique (i.e. if $f(a') = b = f(a)$, then $a' = a$).

To see that $g$ is the inverse of $f$, first consider any $a \in A$. If $f(a) = b$, then $g(b) = a$, by definition of $g$. That is, $g(f(a)) = a$, which shows that $g \circ f = \id_A$. Next, consider any $b \in B$. Then, by definition of $g$, $g(b) = a \in A$ such that $f(a) = b$. That is, $f(g(b)) = b$, which shows that $f \circ g = \id_B$. Thus, $g$ is the inverse of $f$, and therefore $f$ is an isomorphism.

Conversely, suppose that $f \colon A \to B$ is an isomorphism. We will show that $f$ is a bijection. For any $b \in B$, if we take $a = f^{-1}(b)$, then $f(a) = b$, which shows that $f$ is surjective. To see that $f$ is injective, suppose that $a_1, a_2 \in A$ and $f(a_1) = f(a_2)$. Then $a_1 = f^{-1}(f(a_1)) = f^{-1}(f(a_2)) = a_2$. Hence, $f$ is a bijection.
\end{proof}

\subsection{Cardinality of Sets}\label{subsec:Cardinality}
The \newterm{cardinality} of a set is the number of elements in it. As sets can have infinitely many elements, we will define cardinality more rigorously using the concept of set isomorphisms.

\begin{Definition}
Two sets $A$ and $B$ have the same \newterm{cardinality} if $A \cong B$. If $A \cong \{1, \ldots, n\}$, then $A$ is \newterm{finite} of cardinality $n$. If $A$ is not finite, then it is \newterm{infinite}. If $A \cong \mathbb N$, then $A$ is \newterm{countably infinite}. If $A$ is either finite or countably infinite, it is \newterm{countable}. Otherwise, $A$ is \newterm{uncountable} or \newterm{uncountably infinite}.
\end{Definition}

\begin{Exercise}
Prove that $\mathbb Z$ is countably infinite.
\end{Exercise}

\begin{Solution*}
Define a function $f \colon \mathbb Z \to \mathbb N$ by
\begin{equation*}
f(n) = \begin{cases}
2(n + 1), & n \ge 0 \\
2|n| - 1, & n < 0.
\end{cases}
\end{equation*}
Then $f$ is a bijection, which can be shown as follows. Suppose $f(m) = f(n)$, for some $m, n \in \mathbb Z$. Then, either $m, n \ge 0$ and $2(n + 1) = 2(m + 1)$, which implies that $m = n$, or else $m, n < 0$, and $2|m| - 1 = 2|n| - 1$, which implies $|m| = |n$, and hence $m = n$ (since both are negative). Therefore, $f$ is injective. Now, for any $n \in \mathbb N$, $n$ is either even or odd. If it is even, then it is of the form $2k + 2$ for some integer $k \ge 0$, i.e. $n = f(k)$. If it is odd, then it is of the form $2k - 1$, for some $k \ge 1$, i.e. $n = 2|-k| - 1 = f(-k)$. Therefore, $f$ is surjective.

Since there is a bijection from $\mathbb Z$ to $\mathbb N$, we have $\mathbb Z \cong \mathbb N$. Therefore, $\mathbb Z$ is countably infinite.
\end{Solution*}

\begin{Exercise}
Let $A$ and $B$ be two sets. Prove the following.
\begin{enumerate}
\item If $A$ and $B$ are finite, then $A \cup B$ is finite.
\item If $A$ is finite and $B$ is countably infinite, then $A \cup B$ is countably infinite.
\item If $A$ and $B$ are countably infinite, then $A \cup B$ is countably infinite.
\item If $A$ and $B$ are countable, $A \cup B$ is countable.
\end{enumerate}
Hence show that if $A_1, \ldots, A_n$ are $n$ countable sets, and $A = A_1 \cup \cdots \cup A_n$, then
\begin{enumerate}
\item $A$ is countable, and
\item $A$ is countably infinite if and only if at least one $A_i$ is countably infinite.
\end{enumerate}
\end{Exercise}

\begin{Remark*}\label{rem:NBijectionIsEnumeration}
A bijection from $\mathbb N$ to a set $S$ defines an enumeration of all the distinct elements of $S$. That is, suppose that $f \colon \mathbb N \to S$ is a bijection. Now, $f(1)$ is an element of $S$, say $s_1$. Then, $f(2)$ is another element of $S$, say $s_2$, and $s_1 \ne s_2$ (since $f$ is injective). Similarly, $f(3), f(4), \ldots$, are distinct elements $s_3, s_4, \ldots$ of $S$. Since $f$ is surjective, each element of $S$ is $s_n = f(n)$, for some $n \in \mathbb N$. Thus, in order to show that a set is countably infinite, it is sufficient to give an enumeration of all its distinct elements as a sequence. But note that in this case, one must prove that \emph{every} element of the set is indeed an $n$\nth term of the sequence for some positive integer $n$.
\end{Remark*}

\begin{Lemma}\label{lem:CIxCI=CI}
If $A$ and $B$ are countably infinite, then $A \times B$ is countably infinite.
\end{Lemma}

\begin{proof}
Since $A$ and $B$ are countably infinite, there exist bijections $a \colon \mathbb N \to A$ and $b \colon \mathbb N \to B$. For each $n \in \mathbb N$, denote $a(n)$ by $a_n$, and $b(n)$ by $b_n$ (see \cref{rem:NBijectionIsEnumeration}). Then, $A = \{a_1, a_2, \ldots\}$ and $B = \{b_1, b_2, \ldots\}$. Now,
\begin{equation*}
A \times B = \{(a_1, b_1), (a_1, b_2), (a_1, b_3), \ldots, (a_2, b_1), (a_2, b_2), (a_2, b_3), \ldots\}.
\end{equation*}
Consider the elements of $A \times B$ as being arranged on an infinite grid as shown below.
\begin{equation*}
\begin{matrix}
(a_1, b_1) & (a_1, b_2) & (a_1, b_3) & \cdots \\
(a_2, b_1) & (a_2, b_2) & (a_2, b_3) & \cdots \\
(a_3, b_1) & (a_3, b_2) & (a_3, b_3) & \cdots \\
\vdots & \vdots & \vdots & \ddots
\end{matrix}
\end{equation*}
Now, we can define an enumeration of the elements of $A \times B$ by reading the elements along the antidiagonals, starting from $(a_1, b_1)$. Explicitly, the enumeration is
\begin{equation*}
(a_1, b_1),\, (a_1, b_2),\, (a_2, b_1),\, (a_1, b_3),\, (a_2, b_2),\, (a_3, b_1),\, \ldots.
\end{equation*}
Since each antidiagonal is finite, every element $(a_m, b_n)$ appears in the above enumeration after some finite number of terms. Indeed, it is not hard to verify that this enumeration defines a bijection $f \colon A \times B \to \mathbb N$, given by $f(a_m, b_n) = \binom {m + n - 1} 2 + m$.
\end{proof}

From \cref{lem:CIxCI=CI}, it is clear that $\mathbb N \times \mathbb N$, $\mathbb Z \times \mathbb Z$, and $\mathbb Z \times \mathbb N$ are countably infinite. Since each rational number can be written in the form $\frac a b$ where $a \in \mathbb Z$ and $b \in \mathbb N$, the set of rational numbers, it follows that $\mathbb Q$, is countable. Clearly, it is not finite, so in fact, it is countably infinite. 

\begin{Theorem}
The set of rational numbers, $\mathbb Q$, is countably infinite. \hfill $\qed$
\end{Theorem}

\subsubsection{Uncountable Sets}\label{subsec:UncountableSets}
All the examples of infinite sets that we have seen so far were countable. Are there any uncountable sets at all? We shall show that the set of real numbers is uncountably infinite. Recall from \cref{ex:RIsom01} that $\mathbb R \cong (0, 1)$. Thus, in order to show that $\mathbb R$ is uncountable, it is enough to show that $(0, 1)$ is uncountable.

\begin{Theorem}
The open interval $(0, 1)$ consisting of all real numbers strictly between $0$ and $1$ is uncountably infinite.
\end{Theorem}

\begin{proof}
Let $f \colon \mathbb N \to (0, 1))$ be any injection from $\mathbb N$ to $(0, 1)$. Since each real number in the interval $(0, 1)$ has a unique infinite decimal expansion, for each $n \in \mathbb N$, $f(n)$ can be written in the form $f(n) = 0.a_{n1} a_{n2} a_{n3} \cdots$, where $0 \le a_{ni} \le 9$, $i = 1, 2, \ldots$.

Let $x = 0.x_1 x_2 x_3 \ldots$, where
\begin{equation*}
x_n = \begin{cases}
2, & a_{nn} \ne 2 \\
3, & a_{nn} = 2.
\end{cases}
\end{equation*}
Then clearly, $x \in (0, 1)$. But $x \ne f(n)$ for any $n \in \mathbb N$, as can be seen by comparing the $n$\nth digits of $x$ and $f(n)$. If the $n$\nth digit of $f(n)$ is $2$, then the $n$\nth digit of $x$ is $x_n = 3$, whereas if the $n$\nth digit of $f(n)$ is not $2$, then that of $x$ is $x_n = 2$. Thus, $x \notin \im f$, and hence $f$ is not surjective.

The above argument shows that no injection from $\mathbb N$ to $(0, 1)$ can be surjective. That is, there is no bijection (surjective injection) from $\mathbb N$ to $(0, 1)$, and therefore the latter is not countably infinite. Since $(0, 1)$ is infinite, but not countably infinite, it is uncountably infinite.
\end{proof}

\begin{Exercise}
Show that the cardinality of any set is different from that of its power set.
\end{Exercise}

\begin{Solution*}
If $A$ is a finite set of cardinality $n$, then $|2^A| = 2^n \ne n$. Now, suppose that $A$ is an infinite set.

Let $f \colon A \to 2^A$ be an injection. Note that for each $a \in A$, $f(a)$ is an element of $2^A$, i.e. $f(a)$ is a subset of $A$. Define a subset $X$ of $A$ as follows: For each $a \in A$, $a \in X$ if and only if $a \notin f(a)$.

Observe that $X \notin \im f$. For, if $X = f(x)$ for some $x \in A$, then either $x \in X$, in which case $x \notin f(x)$, or $x \notin X$, in which case $x \in f(x)$. But $f(x) = X$, which is a contradiction.

Thus, $f$ is not surjective. Therefore, there is no surjective injection, i.e. bijection from $A$ to $2^A$, and hence $A \not\cong 2^A$.
\end{Solution*}

%\begin{appendices}
%
%\end{appendices}