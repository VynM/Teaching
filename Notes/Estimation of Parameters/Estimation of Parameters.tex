\documentclass[svgnames, a5paper]{article}
\usepackage[top = 20mm, bottom = 15mm, left=10mm, right = 10mm]{geometry}

\usepackage{natbib}

\usepackage{amsmath, amsfonts, amssymb, amsthm}
\setcounter{tocdepth}{3}
\usepackage{physics}

\usepackage{mathtools}
\usepackage{xspace}
\usepackage{enumitem}


\usepackage{tocloft}
\renewcommand{\cftdot}{}

\usepackage{tikz}
\usepackage{hyperref}
\hypersetup{colorlinks, linkcolor = [RGB]{66, 128, 128}, urlcolor = red, linktocpage = true}
\usepackage{bookmark}
\bookmarksetup{color = [RGB]{66, 128, 128}}

\usepackage{hhline, array, multirow}

\usepackage[intoc]{nomencl}
\makenomenclature

\usepackage[toc, page]{appendix}

\usepackage{newpxmath}
\usepackage{charter}
\usepackage[T1]{fontenc}

\usepackage{scrlayer-scrpage}
\ohead{\color{blue!35!black} \scshape VM}
\cfoot*{\pagemark}

\newtheorem{Theorem}{Theorem}[section]
\newtheorem{Lemma}[Theorem]{Lemma}
\newtheorem{Corollary}[Theorem]{Corollary}

\theoremstyle{definition}
\newtheorem{Definition}[Theorem]{Definition}
\newtheorem*{Definition*}{Definition}
\newtheorem{Example}[Theorem]{Example}
\newtheorem*{Example*}{Example}
\newtheorem{Exercise}{Exercise}[section]

\theoremstyle{remark}
\newtheorem*{Remark*}{Remark}
\newtheorem*{Solution*}{Solution}
\newtheorem*{Note*}{Note}

\DeclareMathOperator{\ord}{o}
\newcommand{\Mod}[1]{\ (\mathrm{mod}\ #1)}
\DeclareMathOperator{\lcm}{lcm}

\let\Im\relax
\DeclareMathOperator{\Im}{Im}
\newcommand{\id}{\mathrm{id}}
\renewcommand{\th}{\textsuperscript{th}\xspace}

\newlist{subquests}{enumerate}{2}
\setlist[subquests, 1]{label = (\alph*)}
\setlist[subquests, 2]{label = \roman*.}

\begin{document}

\title{\textbf{Estimation of Parameters}}

\date{}
\maketitle

%\begingroup
%\let\clearpage\relax
%\tableofcontents
%\endgroup

\section{Maximum Likelihood Estimation}\label{sec:MLE}

Let $X$ be a random variable having pdf or pmf $f(x; \theta)$, where $\theta$ is an unknown parameter. The \emph{likelihood function} of $\theta$, corresponding to a sample of size $n$, is
\begin{equation*}
L(\theta) = f(x_1; \theta) f(x_2; \theta) \cdots f(x_n; \theta).
\end{equation*}
The value of $\theta = T(x_1, \ldots, x_n)$ (say) for which $L(\theta)$ is maximum is the \emph{maximum likelihood estimate} of $\theta$. The corresponding statistic $\hat\theta = T(X_1, \ldots, X_n)$ is the \emph{maximum likelihood estimator} (mle) of $\theta$.

Where applicable, we maximise $L(\theta)$ by solving the equation $\dfrac{d}{d\theta} L(\theta) = 0$ for $\theta$. As $L(\theta)$ is the product of $f(x_1; \theta), \ldots, f(x_n; \theta)$, it is more often convenient to maximise the \emph{log-likelihood function} $\log L(\theta) = \sum_{i=1}^{n} \log f(x_i; \theta)$ -- which yields the same estimator $\hat\theta$, since $\log$ is an increasing function.

\begin{Example}
Let $X$ be a random variable with pdf $f(x) = (1 + \theta) x^{\theta}$, $0 < x < 1$, where $\theta > 0$. The likelihood function, assuming a sample of size $n$, will be
\begin{align*}
L(\theta) = 
\end{align*}
\end{Example} 

\section{Confidence Intervals}\label{sec:CI}

\begin{Definition}
If $(X_1, \ldots, X_n)$ is a random sample of size $n$, then the \emph{sample mean} is defined as
\begin{equation*}
\overline X = \dfrac 1 n \sum_{i = 1}^n X_i
\end{equation*}
and the \emph{sample variance} is defined as
\begin{equation*}
S^2 = \dfrac 1 n \sum_{i = 1}^n \pqty{X_i - \overline X}^2.
\end{equation*}
\end{Definition}

\begin{Theorem}
Suppose that $X \sim N(\mu, \sigma^2)$. Then the following hold:
\begin{enumerate}
\item $\dfrac{\overline X - \mu}{\sigma / \sqrt n} \sim N(0, 1)$.
\item $\dfrac{\overline X - \mu}{S/\sqrt{n - 1}} \sim t_{n-1}$.
\item $\dfrac 1 {\sigma^2} \displaystyle\sum_{i=1}^n (X_i - \mu)^2 \sim \chi^2_n$.
\item $\dfrac{nS^2}{\sigma^2} \sim \chi^2_{n - 1}$.
\end{enumerate}
\end{Theorem}

Confidence intervals for the mean and variance of a normal population, under different conditions, are as described in the table given below.
\begin{center}
\def\arraystretch{2.3}
\begin{tabular}{|c|c|>{\setlength{\baselineskip}{2.2\baselineskip}\centering\arraybackslash}p{0.45\textwidth}|}
\hline
\textbf{Parameter} & \textbf{Condition} & \textbf{Interval} \\
\hhline{|===|}
\multirow{2}{*}{$\mu$} & $\sigma^2$ known &
$\pqty{ \overline x - \dfrac{a\sigma}{\sqrt n} , \overline x + \dfrac{a\sigma}{\sqrt n} }$ where $P[-a < Z < a] = p$, $Z \sim N(0, 1)$ \\
\cline{2-3}
 & $\sigma^2$ unknown &
$\pqty{ \overline x - \dfrac{bS}{\sqrt {n-1}} , \overline x + \dfrac{bS}{\sqrt {n-1}} }$ where $P[-b < T < b] = p$, $T \sim t_{n-1}$ \\
\hline
\multirow{2}{*}{$\sigma^2$} & $\mu$ known &
$\pqty{\dfrac 1 b \sum\limits_{i=1}^n (x_i - \mu)^2, \dfrac 1 a \sum\limits_{i=1}^n (x_i - \mu)^2}$ where $P[Y < a] = P[Y > b] = \frac p 2$, $Y \sim \chi^2_n$ \\
\cline{2-3}
 & $\mu$ unknown & $\pqty{ \dfrac{ns^2}{b}, \dfrac{ns^2}{a}}$ where $P[Z < a] = P[Z > b] = \frac p 2$, $Z \sim \chi^2_{n-1}$ \\
\hline
\end{tabular}
\end{center}

\end{document}