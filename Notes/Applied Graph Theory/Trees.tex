\documentclass[svgnames]{amsart}
\usepackage[paperwidth=6in, paperheight=8in, top = 20mm, bottom = 18mm, left=10mm, right = 10mm]{geometry}

\usepackage{amsmath, amsfonts, amssymb, amsthm}
\usepackage{graphicx}
\usepackage{mathtools}
\usepackage{enumitem}

\usepackage[default,light,bold]{sourceserifpro}
\usepackage[T1]{fontenc}
\usepackage{eulervm}

\usepackage{xcolor}
\usepackage{scrlayer-scrpage}
\ohead{\color{blue!35!black} \scshape VM}
\cfoot*{\pagemark}

\renewcommand{\th}{\textsuperscript{th}}
\DeclareMathOperator{\rad}{rad}
\DeclareMathOperator{\ecc}{ecc}

\setlength{\parindent}{0pt}

\newtheorem{Theorem}{Theorem}
\theoremstyle{example}
\theoremstyle{definition}
\newtheorem{Example}{Example}
\newtheorem{Exercise}{Exercise}

\title{Trees}
\date{}

\begin{document}
\maketitle

A \emph{tree} is a connected, acyclic graph. There are several well known characterisations or alternative definitions of trees. We take the given definition as the basic one and prove its equivalence to some others.

\begin{Theorem}\label{thm:UniPath}
A graph $T$ is a tree if and only if there is a unique path joining every two vertices of $T$.
\end{Theorem}
\begin{proof}
First, suppose that $T$ is a tree, and let $u$ and $v$ be vertices of $T$. Since $T$ is connected, there is a path, say $P_1$, joining $u$ and $v$. Now we must show that this path is unique. Assume to the contrary that there exists another path $P_2$ from $u$ to $v$. When traversing $P_1$ from $u$ to $v$, let $w$ be the first vertex that is present on $P_1$ but not $P_2$. Let $x$ be the vertex on $P_1$ preceding $w$, and note that $x$ is on $P_2$ as well. Let $y$ be the next vertex common to both $P_1$ and $P_2$ when traversing $P_1$ from $x$ to $v$. Then the portion of $P_1$ from $x$ to $y$ together with the portion of $P_2$ from $y$ to $x$ forms a cycle in the tree $T$, which is a contradiction. Thus, $P_1$ is the unique path joining $u$ and $v$.

Conversely, suppose that $T$ is a graph in which there is a unique path joining any two vertices. Clearly, $T$ is connected. To show that $T$ is acyclic, suppose that $v_1, v_2, \ldots, v_n$ is a cycle in $T$. Then we get two different paths joining $v_1$ and $v_n$, namely the path $v_1, v_2, \ldots, v_n$ and the path $v_1, v_n$ (since $v_1 \sim v_n$ in the cycle). This contradicts our assumption. Thus, $T$ must be acyclic and hence is a tree.
\end{proof}

The next two results show that the size of a tree is always one less than its order, and that conversely, this property together with either connectedness or acyclicity implies that the graph is a tree.
\begin{Theorem}\label{thm:Conn;p=q+1}
A $(p, q)$-graph $T$ is a tree if and only if it is connected and $p = q + 1$.
\end{Theorem}
\begin{proof}
Let $T$ be a tree with $p$ vertices and $q$ edges. Then $T$ is connected. We prove that $p = q + 1$ by induction. This is clearly true when $p = 1$. Assume it to be true for all trees of order less than $p$. Now in $T$, we know that every two vertices are joined by a unique path. Thus, if $e$ is any edge of $T$, then the graph $T - \{e\}$ obtained by deleting $e$ has exactly two components, say $T_1$ and $T_2$. Each one is a tree, since it is connected and acyclic. Let $T_i$ have $p_i$ vertices and $q_i$ edges, $i = 1, 2$. Then by the hypothesis, $p_i = q_i + 1$ (since $p_i < p$). But $p = p_1 + p_2$ and $q = q_1 + q_2 + 1$ (since the size of $T - \{e\}$ is one less than that of $T$). Thus, $p = q_1 + q_2 + 2 = q + 1$.

For the converse, suppose that $T$ is a connected $(p,q)$-graph with $p = q + 1$. We must show that is acyclic. Suppose to the contrary that $T$ has a cycle $C$ with $k$ vertices. Then $C$ has $k$ edges as well. Since $T$ is connected, there is a path from every vertex not on $C$ to some vertex of $C$. The shortest path from each vertex $v$ not on $C$ to a vertex on $C$ has a unique edge incident with $v$, which is not part of $C$. Since there are $p - k$ vertices in $T$ not on $C$, there are $p - k$ such edges. Thus $q \ge (p - k) + k = p$, which contradicts our assumption that $p = q + 1$. Thus, $T$ must be acyclic.
\end{proof}

In the following theorem, the proof of the direct part is identical to that of Theorem~\ref{thm:Conn;p=q+1}, except for the assertion being about acyclicity rather than connectedness. The proof of the converse part is entirely different.
\begin{Theorem}\label{thm:Acyc;p=q+1}
A $(p,q)$-graph $T$ is a tree if and only if it is acyclic and $p = q + 1$.
\end{Theorem}
\begin{proof}
Let $T$ be a tree with $p$ vertices and $q$ edges. Then $T$ is acyclic. We prove that $p = q + 1$ by induction. This is clearly true when $p = 1$. Assume it to be true for all trees of order less than $p$. Now in $T$, we know that every two vertices are joined by a unique path. Thus, if $e$ is any edge of $T$, then the graph $T - \{e\}$ obtained by deleting $e$ has exactly two components, say $T_1$ and $T_2$. Each one is a tree, since it is connected and acyclic. Let $T_i$ have $p_i$ vertices and $q_i$ edges, $i = 1, 2$. Then by the hypothesis, $p_i = q_i + 1$ (since $p_i < p$). But $p = p_1 + p_2$ and $q = q_1 + q_2 + 1$ (since the size of $T - \{e\}$ is one less than that of $T$). Thus, $p = q_1 + q_2 + 2 = q + 1$.

Conversely, suppose that $T$ is an acyclic $(p,q)$-graph with $p = q + 1$. To show that $T$ is connected, we need to prove that it is connected -- i.e., it has only one component. Let $T$ have $k$ components $T_1, \ldots, T_k$. Each one is acyclic, and being connected, is a tree. Thus from the first part of the theorem, we know that if $p_i$ and $q_i$ are respectively the order and size of the component $T_i$, $p_i = q_i + 1$. Now $p = p_1 + \cdots p_k = (q_1 + 1) + \cdots (q_k + 1) = q + k$. But we know that $p = q + 1$. Therefore, $k = 1$. Thus, $T$ is a tree.
\end{proof}

\begin{Exercise}
A \emph{pendant vertex} of a graph is a vertex of degree $1$. Prove that every non-trivial tree contains at least two pendant vertices.\\
{\tiny \textbf{Hint:} Observe that a non-trivial tree cannot have a vertex of degree zero. Use Handshaking Lemma and assume every degree is at least $2$ to get a contradiction.}
\end{Exercise}

\begin{Exercise}
The \emph{centre} of a graph $G$ is the set of all vertices of $G$ with minimum eccentricity -- i.e., the set of all vertices $v$ of $G$ with $\ecc v = \rad v$. Show that every tree has a centre consisting of either exactly one vertex or exactly two adjacent vertices.\\
{\tiny \textbf{Hint:} Observe that deleting all pendant vertices of a tree results in a new tree with the same centre.}
\end{Exercise}
\end{document}